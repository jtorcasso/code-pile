
\documentclass{article}

\usepackage[margin=0.5in]{geometry}
\usepackage{fancyvrb}
\usepackage{color}
\usepackage{framed}


\makeatletter
\def\PY@reset{\let\PY@it=\relax \let\PY@bf=\relax%
    \let\PY@ul=\relax \let\PY@tc=\relax%
    \let\PY@bc=\relax \let\PY@ff=\relax}
\def\PY@tok#1{\csname PY@tok@#1\endcsname}
\def\PY@toks#1+{\ifx\relax#1\empty\else%
    \PY@tok{#1}\expandafter\PY@toks\fi}
\def\PY@do#1{\PY@bc{\PY@tc{\PY@ul{%
    \PY@it{\PY@bf{\PY@ff{#1}}}}}}}
\def\PY#1#2{\PY@reset\PY@toks#1+\relax+\PY@do{#2}}

\expandafter\def\csname PY@tok@gd\endcsname{\def\PY@tc##1{\textcolor[rgb]{0.63,0.00,0.00}{##1}}}
\expandafter\def\csname PY@tok@gu\endcsname{\let\PY@bf=\textbf\def\PY@tc##1{\textcolor[rgb]{0.50,0.00,0.50}{##1}}}
\expandafter\def\csname PY@tok@gt\endcsname{\def\PY@tc##1{\textcolor[rgb]{0.00,0.27,0.87}{##1}}}
\expandafter\def\csname PY@tok@gs\endcsname{\let\PY@bf=\textbf}
\expandafter\def\csname PY@tok@gr\endcsname{\def\PY@tc##1{\textcolor[rgb]{1.00,0.00,0.00}{##1}}}
\expandafter\def\csname PY@tok@cm\endcsname{\let\PY@it=\textit\def\PY@tc##1{\textcolor[rgb]{0.25,0.50,0.50}{##1}}}
\expandafter\def\csname PY@tok@vg\endcsname{\def\PY@tc##1{\textcolor[rgb]{0.10,0.09,0.49}{##1}}}
\expandafter\def\csname PY@tok@m\endcsname{\def\PY@tc##1{\textcolor[rgb]{0.40,0.40,0.40}{##1}}}
\expandafter\def\csname PY@tok@mh\endcsname{\def\PY@tc##1{\textcolor[rgb]{0.40,0.40,0.40}{##1}}}
\expandafter\def\csname PY@tok@go\endcsname{\def\PY@tc##1{\textcolor[rgb]{0.53,0.53,0.53}{##1}}}
\expandafter\def\csname PY@tok@ge\endcsname{\let\PY@it=\textit}
\expandafter\def\csname PY@tok@vc\endcsname{\def\PY@tc##1{\textcolor[rgb]{0.10,0.09,0.49}{##1}}}
\expandafter\def\csname PY@tok@il\endcsname{\def\PY@tc##1{\textcolor[rgb]{0.40,0.40,0.40}{##1}}}
\expandafter\def\csname PY@tok@cs\endcsname{\let\PY@it=\textit\def\PY@tc##1{\textcolor[rgb]{0.25,0.50,0.50}{##1}}}
\expandafter\def\csname PY@tok@cp\endcsname{\def\PY@tc##1{\textcolor[rgb]{0.74,0.48,0.00}{##1}}}
\expandafter\def\csname PY@tok@gi\endcsname{\def\PY@tc##1{\textcolor[rgb]{0.00,0.63,0.00}{##1}}}
\expandafter\def\csname PY@tok@gh\endcsname{\let\PY@bf=\textbf\def\PY@tc##1{\textcolor[rgb]{0.00,0.00,0.50}{##1}}}
\expandafter\def\csname PY@tok@ni\endcsname{\let\PY@bf=\textbf\def\PY@tc##1{\textcolor[rgb]{0.60,0.60,0.60}{##1}}}
\expandafter\def\csname PY@tok@nl\endcsname{\def\PY@tc##1{\textcolor[rgb]{0.63,0.63,0.00}{##1}}}
\expandafter\def\csname PY@tok@nn\endcsname{\let\PY@bf=\textbf\def\PY@tc##1{\textcolor[rgb]{0.00,0.00,1.00}{##1}}}
\expandafter\def\csname PY@tok@no\endcsname{\def\PY@tc##1{\textcolor[rgb]{0.53,0.00,0.00}{##1}}}
\expandafter\def\csname PY@tok@na\endcsname{\def\PY@tc##1{\textcolor[rgb]{0.49,0.56,0.16}{##1}}}
\expandafter\def\csname PY@tok@nb\endcsname{\def\PY@tc##1{\textcolor[rgb]{0.00,0.50,0.00}{##1}}}
\expandafter\def\csname PY@tok@nc\endcsname{\let\PY@bf=\textbf\def\PY@tc##1{\textcolor[rgb]{0.00,0.00,1.00}{##1}}}
\expandafter\def\csname PY@tok@nd\endcsname{\def\PY@tc##1{\textcolor[rgb]{0.67,0.13,1.00}{##1}}}
\expandafter\def\csname PY@tok@ne\endcsname{\let\PY@bf=\textbf\def\PY@tc##1{\textcolor[rgb]{0.82,0.25,0.23}{##1}}}
\expandafter\def\csname PY@tok@nf\endcsname{\def\PY@tc##1{\textcolor[rgb]{0.00,0.00,1.00}{##1}}}
\expandafter\def\csname PY@tok@si\endcsname{\let\PY@bf=\textbf\def\PY@tc##1{\textcolor[rgb]{0.73,0.40,0.53}{##1}}}
\expandafter\def\csname PY@tok@s2\endcsname{\def\PY@tc##1{\textcolor[rgb]{0.73,0.13,0.13}{##1}}}
\expandafter\def\csname PY@tok@vi\endcsname{\def\PY@tc##1{\textcolor[rgb]{0.10,0.09,0.49}{##1}}}
\expandafter\def\csname PY@tok@nt\endcsname{\let\PY@bf=\textbf\def\PY@tc##1{\textcolor[rgb]{0.00,0.50,0.00}{##1}}}
\expandafter\def\csname PY@tok@nv\endcsname{\def\PY@tc##1{\textcolor[rgb]{0.10,0.09,0.49}{##1}}}
\expandafter\def\csname PY@tok@s1\endcsname{\def\PY@tc##1{\textcolor[rgb]{0.73,0.13,0.13}{##1}}}
\expandafter\def\csname PY@tok@sh\endcsname{\def\PY@tc##1{\textcolor[rgb]{0.73,0.13,0.13}{##1}}}
\expandafter\def\csname PY@tok@sc\endcsname{\def\PY@tc##1{\textcolor[rgb]{0.73,0.13,0.13}{##1}}}
\expandafter\def\csname PY@tok@sx\endcsname{\def\PY@tc##1{\textcolor[rgb]{0.00,0.50,0.00}{##1}}}
\expandafter\def\csname PY@tok@bp\endcsname{\def\PY@tc##1{\textcolor[rgb]{0.00,0.50,0.00}{##1}}}
\expandafter\def\csname PY@tok@c1\endcsname{\let\PY@it=\textit\def\PY@tc##1{\textcolor[rgb]{0.25,0.50,0.50}{##1}}}
\expandafter\def\csname PY@tok@kc\endcsname{\let\PY@bf=\textbf\def\PY@tc##1{\textcolor[rgb]{0.00,0.50,0.00}{##1}}}
\expandafter\def\csname PY@tok@c\endcsname{\let\PY@it=\textit\def\PY@tc##1{\textcolor[rgb]{0.25,0.50,0.50}{##1}}}
\expandafter\def\csname PY@tok@mf\endcsname{\def\PY@tc##1{\textcolor[rgb]{0.40,0.40,0.40}{##1}}}
\expandafter\def\csname PY@tok@err\endcsname{\def\PY@bc##1{\setlength{\fboxsep}{0pt}\fcolorbox[rgb]{1.00,0.00,0.00}{1,1,1}{\strut ##1}}}
\expandafter\def\csname PY@tok@kd\endcsname{\let\PY@bf=\textbf\def\PY@tc##1{\textcolor[rgb]{0.00,0.50,0.00}{##1}}}
\expandafter\def\csname PY@tok@ss\endcsname{\def\PY@tc##1{\textcolor[rgb]{0.10,0.09,0.49}{##1}}}
\expandafter\def\csname PY@tok@sr\endcsname{\def\PY@tc##1{\textcolor[rgb]{0.73,0.40,0.53}{##1}}}
\expandafter\def\csname PY@tok@mo\endcsname{\def\PY@tc##1{\textcolor[rgb]{0.40,0.40,0.40}{##1}}}
\expandafter\def\csname PY@tok@kn\endcsname{\let\PY@bf=\textbf\def\PY@tc##1{\textcolor[rgb]{0.00,0.50,0.00}{##1}}}
\expandafter\def\csname PY@tok@mi\endcsname{\def\PY@tc##1{\textcolor[rgb]{0.40,0.40,0.40}{##1}}}
\expandafter\def\csname PY@tok@gp\endcsname{\let\PY@bf=\textbf\def\PY@tc##1{\textcolor[rgb]{0.00,0.00,0.50}{##1}}}
\expandafter\def\csname PY@tok@o\endcsname{\def\PY@tc##1{\textcolor[rgb]{0.40,0.40,0.40}{##1}}}
\expandafter\def\csname PY@tok@kr\endcsname{\let\PY@bf=\textbf\def\PY@tc##1{\textcolor[rgb]{0.00,0.50,0.00}{##1}}}
\expandafter\def\csname PY@tok@s\endcsname{\def\PY@tc##1{\textcolor[rgb]{0.73,0.13,0.13}{##1}}}
\expandafter\def\csname PY@tok@kp\endcsname{\def\PY@tc##1{\textcolor[rgb]{0.00,0.50,0.00}{##1}}}
\expandafter\def\csname PY@tok@w\endcsname{\def\PY@tc##1{\textcolor[rgb]{0.73,0.73,0.73}{##1}}}
\expandafter\def\csname PY@tok@kt\endcsname{\def\PY@tc##1{\textcolor[rgb]{0.69,0.00,0.25}{##1}}}
\expandafter\def\csname PY@tok@ow\endcsname{\let\PY@bf=\textbf\def\PY@tc##1{\textcolor[rgb]{0.67,0.13,1.00}{##1}}}
\expandafter\def\csname PY@tok@sb\endcsname{\def\PY@tc##1{\textcolor[rgb]{0.73,0.13,0.13}{##1}}}
\expandafter\def\csname PY@tok@k\endcsname{\let\PY@bf=\textbf\def\PY@tc##1{\textcolor[rgb]{0.00,0.50,0.00}{##1}}}
\expandafter\def\csname PY@tok@se\endcsname{\let\PY@bf=\textbf\def\PY@tc##1{\textcolor[rgb]{0.73,0.40,0.13}{##1}}}
\expandafter\def\csname PY@tok@sd\endcsname{\let\PY@it=\textit\def\PY@tc##1{\textcolor[rgb]{0.73,0.13,0.13}{##1}}}

\def\PYZbs{\char`\\}
\def\PYZus{\char`\_}
\def\PYZob{\char`\{}
\def\PYZcb{\char`\}}
\def\PYZca{\char`\^}
\def\PYZam{\char`\&}
\def\PYZlt{\char`\<}
\def\PYZgt{\char`\>}
\def\PYZsh{\char`\#}
\def\PYZpc{\char`\%}
\def\PYZdl{\char`\$}
\def\PYZhy{\char`\-}
\def\PYZsq{\char`\'}
\def\PYZdq{\char`\"}
\def\PYZti{\char`\~}
% for compatibility with earlier versions
\def\PYZat{@}
\def\PYZlb{[}
\def\PYZrb{]}
\makeatother


\begin{document}

\section*{Motivation}

The research process is a collection of tasks, sometimes performed by multiple 
people, which combine, in the end, into a carefully structured, defensible 
argument. Each task might require a different tool or person, and the 
order in which the tasks take place often matters. Hence, a more formal 
and well-planned research project combines the disparate tasks efficiently. 
Part of this efficiency means the project's results are also reproducible, 
and, as a result, more defensible. 

Typical Tasks of the Empirical Researcher:

\begin{itemize}
\item retrieve/convert data
\item clean/organize data
\item run analyses
\item generate figures and tables
\item cater discussion and results to the empirical evidence
\end{itemize}

We should all be familiar to at least some of these tasks. Often different
tools and people work optimally at different links in the chain. For instance, 
you might use STATA to clean data, MATLAB to run analyses, Python to 
generate figures and tables, and LateX to compile the containing paper. 
Often you have research assistants cleaning the data, and PhDs running analyses and 
discussing the results. You might be passing data back and forth, funneling 
everything through different machines with different configurations, operating 
systems, and installed software. The advantage of doing it this way is that each 
program or person has a functional advantage at each leg of the project's journey. 
However, all of this movement can increase the chances of error and slow your project
down.

In this tutorial, I plan on showing how you can use Python to manage your workflow
efficiently. The concepts we will cover include:

\begin{enumerate}
\item Understanding how Python works with your file system
\item Integrating Python with STATA, R, and LateX
\item How to design your own modules for projects
\item How to organize your project to use Python effectively
\item Getting to know some of Python's useful packages for research
\item Version Control
\item waf and wscript
\end{enumerate}
\noindent \section*{How Python interacts with your file system}
\begin{quote} 
``I want it now! The meatloaf! 
I never know what she's doing back there.''
\end{quote}
Ok, so how does Python bring you your meatloaf? And where are all of 
the ingredients in the kitchen? Before integrating Python into your
workflow you need to know where it gets packages and where it looks
and writes files, and how to use these packages and files in your code.
Let's find out where Python is looking on 
your system for modules and packages. One place it looks is wherever 
your environment variable, PYTHONPATH, points to. Let's print all the 
paths in the PYTHONPATH using the os package. 

\begin{framed}
\begin{Verbatim}[commandchars=\\\{\}]
\PY{k+kn}{import} \PY{n+nn}{os}
\PY{n}{paths} \PY{o}{=} \PY{n}{os}\PY{o}{.}\PY{n}{environ}\PY{p}{[}\PY{l+s}{\PYZsq{}}\PY{l+s}{PYTHONPATH}\PY{l+s}{\PYZsq{}}\PY{p}{]}\PY{o}{.}\PY{n}{split}\PY{p}{(}\PY{l+s}{\PYZsq{}}\PY{l+s}{:}\PY{l+s}{\PYZsq{}}\PY{p}{)}
\PY{k}{for} \PY{n}{path} \PY{o+ow}{in} \PY{n}{paths}\PY{p}{:}
    \PY{k}{print} \PY{n}{path}
\end{Verbatim}
\end{framed}
    
\noindent If you experienced a keyerror, have no fear, it just means this variable 
is not set. There are other places Python looks for packages.   
It will also look in your current working directory. 

\begin{framed}
\begin{Verbatim}[commandchars=\\\{\}]
\PY{k}{print} \PY{n}{os}\PY{o}{.}\PY{n}{getcwd}\PY{p}{(}\PY{p}{)}
\end{Verbatim}
\end{framed}
    
\noindent Let's try to import a module we have made. Saving the following code as `test\_module.py'
in the directory you are using for this tutorial file:

\begin{framed}
\begin{Verbatim}[commandchars=\\\{\}]
\PY{k}{def} \PY{n+nf}{crazyWord}\PY{p}{(}\PY{n}{string}\PY{p}{)}\PY{p}{:}
    
    \PY{k}{return} \PY{n}{string} \PY{o}{+} \PY{l+s}{\PYZsq{}}\PY{l+s}{, now that is a crazy word}\PY{l+s}{\PYZsq{}}
\end{Verbatim}
\end{framed}
    
\noindent Let's try importing the module first with a different directory and then using the current directory

\begin{framed}
\begin{Verbatim}[commandchars=\\\{\}]
\PY{k}{try}\PY{p}{:}
    \PY{n}{current\PYZus{}directory} \PY{o}{=} \PY{n}{os}\PY{o}{.}\PY{n}{getcwd}\PY{p}{(}\PY{p}{)}
    \PY{n}{os}\PY{o}{.}\PY{n}{chdir}\PY{p}{(}\PY{l+s}{\PYZdq{}}\PY{l+s}{..}\PY{l+s}{\PYZbs{}}\PY{l+s}{..}\PY{l+s}{\PYZdq{}}\PY{p}{)}
    \PY{k+kn}{import} \PY{n+nn}{test\PYZus{}module} \PY{k+kn}{as} \PY{n+nn}{testing}
\PY{k}{except}\PY{p}{:}
    \PY{k}{print} \PY{l+s}{\PYZdq{}}\PY{l+s}{Failed to find module in first directory}\PY{l+s}{\PYZdq{}}
    \PY{n}{os}\PY{o}{.}\PY{n}{chdir}\PY{p}{(}\PY{n}{current\PYZus{}directory}\PY{p}{)}
    \PY{k+kn}{import} \PY{n+nn}{test\PYZus{}module} \PY{k+kn}{as} \PY{n+nn}{testing}
    \PY{k}{print} \PY{l+s}{\PYZdq{}}\PY{l+s}{Succeeded at finding module in }\PY{l+s}{\PYZdq{}} \PY{o}{+} \PY{n}{os}\PY{o}{.}\PY{n}{getcwd}\PY{p}{(}\PY{p}{)}
\PY{k}{print} \PY{n}{testing}\PY{o}{.}\PY{n}{crazyWord}\PY{p}{(}\PY{l+s}{\PYZsq{}}\PY{l+s}{hippopotamus}\PY{l+s}{\PYZsq{}}\PY{p}{)}
\end{Verbatim}
\end{framed}
    
\noindent Ok, so Python could not find the module when you changed the current working directory
but after changing it back, it found it just fine. Python looks for modules in a certain
order: first it looks in the current working directory, then it looks 
in the paths contained in the environment variable PYTHONPATH, and then it looks for 
them in the location in which your initial installation put packages and modules (where
the standard library packages are). You can see where this original installation-dependent
path is with the following:

\begin{framed}
\begin{Verbatim}[commandchars=\\\{\}]
\PY{k+kn}{import} \PY{n+nn}{sys}
\PY{k}{for} \PY{n}{path} \PY{o+ow}{in} \PY{n}{sys}\PY{o}{.}\PY{n}{path}\PY{p}{:}
    \PY{k}{print} \PY{n}{path}
    
\end{Verbatim}
\end{framed}
    
\noindent Now what about files? Python looks for files in the current working directory. Let's try
the following: first we create a file, then we save, and then we open it back up. 

\begin{framed}
\begin{Verbatim}[commandchars=\\\{\}]
\PY{n}{new\PYZus{}file} \PY{o}{=} \PY{n+nb}{open}\PY{p}{(}\PY{l+s}{\PYZsq{}}\PY{l+s}{a\PYZus{}text\PYZus{}file.txt}\PY{l+s}{\PYZsq{}}\PY{p}{,} \PY{l+s}{\PYZsq{}}\PY{l+s}{w}\PY{l+s}{\PYZsq{}}\PY{p}{)}
\PY{n}{important\PYZus{}message} \PY{o}{=} \PY{l+s}{\PYZsq{}}\PY{l+s}{John Adams and Thomas Jefferson both died on July 4th, 1826}\PY{l+s}{\PYZsq{}}
\PY{n}{new\PYZus{}file}\PY{o}{.}\PY{n}{write}\PY{p}{(}\PY{n}{important\PYZus{}message}\PY{p}{)}
\PY{n}{new\PYZus{}file}\PY{o}{.}\PY{n}{close}\PY{p}{(}\PY{p}{)}
\PY{k}{print} \PY{n}{os}\PY{o}{.}\PY{n}{getcwd}\PY{p}{(}\PY{p}{)}
\end{Verbatim}
\end{framed}
    
\noindent Let's look at the files in the current directory.

\begin{framed}
\begin{Verbatim}[commandchars=\\\{\}]
\PY{k}{for} \PY{n}{filename} \PY{o+ow}{in} \PY{n}{os}\PY{o}{.}\PY{n}{listdir}\PY{p}{(}\PY{n}{os}\PY{o}{.}\PY{n}{getcwd}\PY{p}{(}\PY{p}{)}\PY{p}{)}\PY{p}{:}
    \PY{k}{print} \PY{n}{filename}
\end{Verbatim}
\end{framed}
    
\noindent You see the file we just created is in there. Open it. 

\begin{framed}
\begin{Verbatim}[commandchars=\\\{\}]
\PY{n}{old\PYZus{}file} \PY{o}{=} \PY{n+nb}{open}\PY{p}{(}\PY{l+s}{\PYZsq{}}\PY{l+s}{a\PYZus{}text\PYZus{}file.txt}\PY{l+s}{\PYZsq{}}\PY{p}{,} \PY{l+s}{\PYZsq{}}\PY{l+s}{r}\PY{l+s}{\PYZsq{}}\PY{p}{)}
\end{Verbatim}
\end{framed}
    
\noindent Read the text contained in the file

\begin{framed}
\begin{Verbatim}[commandchars=\\\{\}]
\PY{n}{text\PYZus{}from\PYZus{}file} \PY{o}{=} \PY{n}{old\PYZus{}file}\PY{o}{.}\PY{n}{readlines}\PY{p}{(}\PY{p}{)}
\PY{k}{print} \PY{n}{text\PYZus{}from\PYZus{}file}
\end{Verbatim}
\end{framed}
    
\noindent Why is all of this important? Because what we want to integrate
outside scripts and datasets into our projects, we want to 
be able to tie together modules or scripts from other programming
languages and we must understand how Python will find and work with these
files. Thus far, we learned how Python looks for outside files and packages, 
how to write and read files, and how to create and use a Python module of 
our own. 
\section*{Running Code from Other Languages in Python}
First of all, why do we even continue to use other languages if Python is so great? 
Several reasons: (i) other languages might have syntactical advantages 
over Python (STATA for data cleaning as an example), (ii) Python might 
not have good packages for some types of analyses, and (iii) not everybody
knows Python, but if you work with people who use STATA primarily you can still
integrate their code into the project. Let's see how it's done. 
The first method uses the terminal or command line. Let's create a stata file. 

\begin{framed}
\begin{Verbatim}[commandchars=\\\{\}]
\PY{n}{stata\PYZus{}code} \PY{o}{=} \PY{l+s}{\PYZsq{}\PYZsq{}\PYZsq{}}
\PY{l+s}{clear all}
\PY{l+s}{set obs 5}
\PY{l+s}{gen column1 = 3}
\PY{l+s}{gen column2 = 4}
\PY{l+s}{outsheet using fake\PYZus{}dataset.csv, comma replace}
\PY{l+s}{clear all}
\PY{l+s}{\PYZsq{}\PYZsq{}\PYZsq{}}
\PY{n}{stata\PYZus{}file} \PY{o}{=} \PY{n+nb}{open}\PY{p}{(}\PY{l+s}{\PYZsq{}}\PY{l+s}{fake\PYZus{}statafile.do}\PY{l+s}{\PYZsq{}}\PY{p}{,} \PY{l+s}{\PYZsq{}}\PY{l+s}{w}\PY{l+s}{\PYZsq{}}\PY{p}{)}
\PY{n}{stata\PYZus{}file}\PY{o}{.}\PY{n}{write}\PY{p}{(}\PY{n}{stata\PYZus{}code}\PY{p}{)}
\PY{n}{stata\PYZus{}file}\PY{o}{.}\PY{n}{close}\PY{p}{(}\PY{p}{)}
\end{Verbatim}
\end{framed}
    
\noindent And let's run the stata code from Python

\begin{framed}
\begin{Verbatim}[commandchars=\\\{\}]
\PY{k+kn}{import} \PY{n+nn}{subprocess}
\PY{c}{\PYZsh{}Path to the executable (your path is probably different)}
\PY{c}{\PYZsh{}stata\PYZus{}executable\PYZus{}path = r\PYZsq{}/usr/local/stata12/stata\PYZhy{}mp\PYZsq{}}
\PY{c}{\PYZsh{}Call STATA through command line}
\PY{c}{\PYZsh{}subprocess.call([stata\PYZus{}executable\PYZus{}path, r\PYZsq{}/e\PYZsq{}, \PYZsq{}do\PYZsq{}, \PYZsq{}fake\PYZus{}statafile.do\PYZsq{}])}
\end{Verbatim}
\end{framed}
    
\noindent If the call worked, you should see a dataset in your current directory
called fake\_dataset.csv

\begin{framed}
\begin{Verbatim}[commandchars=\\\{\}]
\PY{k}{for} \PY{n}{filename} \PY{o+ow}{in} \PY{n}{os}\PY{o}{.}\PY{n}{listdir}\PY{p}{(}\PY{n}{os}\PY{o}{.}\PY{n}{getcwd}\PY{p}{(}\PY{p}{)}\PY{p}{)}\PY{p}{:}
    \PY{k}{print} \PY{n}{filename}
\end{Verbatim}
\end{framed}
    
\noindent \section*{The Power of Python}
Python is so powerful because of how it fosters community and cooperation. 
The community serves not only as a potent source of information but a source
of tried and tested tools. We call these tools third party packages. And no, 
they aren't the shady dealings of the underworld, they are written by talented
programmers eager to share the fruits of their labor with the world. Consider 
two popular packages for data analysis: Numpy and Pandas

\paragraph{Numpy} Numpy or numerical Python works with arrays and is great for
raw data. It's libraries were written in optimally in C, so it's super fast. 

\paragraph{Pandas} Pandas has all the dataset functionality as R. In fact, it 
was modeled off of R and uses the dataframe. It provides functionality for 
SQL-like merges and is great to partner with Numpy. Typically, I work 
at the higher-level in Pandas and then when I run analyses I convert
dataframes into arrays. First let's create a dataset to work with

\begin{framed}
\begin{Verbatim}[commandchars=\\\{\}]
\PY{n}{fake\PYZus{}csvtext} \PY{o}{=} \PY{l+s}{\PYZsq{}\PYZsq{}\PYZsq{}}\PY{l+s}{id,column1,column2,column3}
\PY{l+s}{1,10203.0,49495.494,\PYZhy{}9999}
\PY{l+s}{2,3242.44,4404.405,333304.454}
\PY{l+s}{3,22323,454545,454.44}
\PY{l+s}{4,\PYZhy{}9999,4944959,454.54}
\PY{l+s}{5,1123.123,34343.49,454.454}
\PY{l+s}{6,49923.3434,99585.454,23230.445}
\PY{l+s}{\PYZsq{}\PYZsq{}\PYZsq{}}
\PY{n}{fake\PYZus{}csvfile} \PY{o}{=} \PY{n+nb}{open}\PY{p}{(}\PY{l+s}{\PYZsq{}}\PY{l+s}{fake\PYZus{}csvfile.csv}\PY{l+s}{\PYZsq{}}\PY{p}{,} \PY{l+s}{\PYZsq{}}\PY{l+s}{w}\PY{l+s}{\PYZsq{}}\PY{p}{)}
\PY{n}{fake\PYZus{}csvfile}\PY{o}{.}\PY{n}{write}\PY{p}{(}\PY{n}{fake\PYZus{}csvtext}\PY{p}{)}
\PY{n}{fake\PYZus{}csvfile}\PY{o}{.}\PY{n}{close}\PY{p}{(}\PY{p}{)}
\end{Verbatim}
\end{framed}
    
\noindent Now we can work with our dataset in Pandas and Numpy

\begin{framed}
\begin{Verbatim}[commandchars=\\\{\}]
\PY{k+kn}{import} \PY{n+nn}{pandas} \PY{k+kn}{as} \PY{n+nn}{pd}
\PY{k+kn}{import} \PY{n+nn}{numpy} \PY{k+kn}{as} \PY{n+nn}{np}
\PY{n}{data} \PY{o}{=} \PY{n}{pd}\PY{o}{.}\PY{n}{read\PYZus{}csv}\PY{p}{(}\PY{l+s}{\PYZsq{}}\PY{l+s}{fake\PYZus{}csvfile.csv}\PY{l+s}{\PYZsq{}}\PY{p}{,} \PY{n}{index\PYZus{}col} \PY{o}{=} \PY{l+m+mi}{0}\PY{p}{,} \PY{n}{na\PYZus{}values} \PY{o}{=} \PY{o}{\PYZhy{}}\PY{l+m+mi}{9999}\PY{p}{)}
\PY{k}{print} \PY{n}{data}
\PY{k}{print} \PY{n}{data}\PY{p}{[}\PY{l+s}{\PYZsq{}}\PY{l+s}{column1}\PY{l+s}{\PYZsq{}}\PY{p}{]}\PY{o}{.}\PY{n}{mean}\PY{p}{(}\PY{p}{)}
\PY{k}{print} \PY{n}{data}\PY{p}{[}\PY{p}{[}\PY{l+s}{\PYZsq{}}\PY{l+s}{column1}\PY{l+s}{\PYZsq{}}\PY{p}{,} \PY{l+s}{\PYZsq{}}\PY{l+s}{column3}\PY{l+s}{\PYZsq{}}\PY{p}{]}\PY{p}{]}
\end{Verbatim}
\end{framed}
    
\noindent Now let's take the data from pandas and use numpy with statsmodels to quickly run analyses

\begin{framed}
\begin{Verbatim}[commandchars=\\\{\}]
\PY{n}{nonmissing\PYZus{}data} \PY{o}{=} \PY{n}{data}\PY{p}{[}\PY{o}{\PYZti{}}\PY{n}{np}\PY{o}{.}\PY{n}{isnan}\PY{p}{(}\PY{n}{data}\PY{p}{)}\PY{o}{.}\PY{n}{any}\PY{p}{(}\PY{n}{axis}\PY{o}{=}\PY{l+m+mi}{1}\PY{p}{)}\PY{p}{]}
\PY{n}{numpy\PYZus{}array} \PY{o}{=} \PY{n}{nonmissing\PYZus{}data}\PY{o}{.}\PY{n}{as\PYZus{}matrix}\PY{p}{(}\PY{p}{)}
\PY{k}{print} \PY{n}{numpy\PYZus{}array}
\PY{k+kn}{import} \PY{n+nn}{statsmodels.api} \PY{k+kn}{as} \PY{n+nn}{sm}
\PY{n}{Y} \PY{o}{=} \PY{n}{numpy\PYZus{}array}\PY{p}{[}\PY{p}{:}\PY{p}{,}\PY{l+m+mi}{0}\PY{p}{]}
\PY{n}{X} \PY{o}{=} \PY{n}{numpy\PYZus{}array}\PY{p}{[}\PY{p}{:}\PY{p}{,}\PY{p}{[}\PY{l+m+mi}{1}\PY{p}{,}\PY{l+m+mi}{2}\PY{p}{]}\PY{p}{]}
\PY{k}{print} \PY{n}{np}\PY{o}{.}\PY{n}{hstack}\PY{o}{.}\PY{n}{\PYZus{}\PYZus{}doc\PYZus{}\PYZus{}}
\PY{n}{X} \PY{o}{=} \PY{n}{np}\PY{o}{.}\PY{n}{hstack}\PY{p}{(}\PY{p}{(}\PY{n}{np}\PY{o}{.}\PY{n}{ones}\PY{p}{(}\PY{p}{(}\PY{n+nb}{len}\PY{p}{(}\PY{n}{X}\PY{p}{)}\PY{p}{,}\PY{l+m+mi}{1}\PY{p}{)}\PY{p}{)}\PY{p}{,} \PY{n}{X}\PY{p}{)}\PY{p}{)}
\PY{k}{print} \PY{n}{sm}\PY{o}{.}\PY{n}{OLS}\PY{o}{.}\PY{n}{\PYZus{}\PYZus{}doc\PYZus{}\PYZus{}}
\PY{n}{model} \PY{o}{=} \PY{n}{sm}\PY{o}{.}\PY{n}{OLS}\PY{p}{(}\PY{n}{Y}\PY{p}{,} \PY{n}{X}\PY{p}{)}
\PY{n}{results} \PY{o}{=} \PY{n}{model}\PY{o}{.}\PY{n}{fit}\PY{p}{(}\PY{p}{)}
\PY{k}{print} \PY{n}{sm}\PY{o}{.}\PY{n}{regression}\PY{o}{.}\PY{n}{linear\PYZus{}model}\PY{o}{.}\PY{n}{RegressionResults}\PY{o}{.}\PY{n}{\PYZus{}\PYZus{}doc\PYZus{}\PYZus{}}
\PY{k}{print} \PY{n}{results}\PY{o}{.}\PY{n}{params}
\PY{k}{print} \PY{n}{results}\PY{o}{.}\PY{n}{resid}
\PY{k}{print} \PY{n}{results}\PY{o}{.}\PY{n}{rsquared\PYZus{}adj}
\PY{k}{print} \PY{n}{results}\PY{o}{.}\PY{n}{pvalues}
\PY{k}{print} \PY{n}{results}\PY{o}{.}\PY{n}{fittedvalues}
\end{Verbatim}
\end{framed}
    
\end{document}