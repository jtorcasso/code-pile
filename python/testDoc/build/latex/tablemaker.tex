% Generated by Sphinx.
\def\sphinxdocclass{report}
\documentclass[letterpaper,10pt,english]{sphinxmanual}
\usepackage[utf8]{inputenc}
\DeclareUnicodeCharacter{00A0}{\nobreakspace}
\usepackage[T1]{fontenc}
\usepackage{babel}
\usepackage{times}
\usepackage[Bjarne]{fncychap}
\usepackage{longtable}
\usepackage{sphinx}
\usepackage{multirow}

\usepackage{booktabs}
\usepackage{makecell}
\usepackage{graphicx}
\usepackage{fontenc}
\usepackage{longtable}
\usepackage{rotating}
\usepackage{multirow}

\renewcommand{\rothead}[2][60]{\makebox[9mm][c]{\rotatebox{#1}{\makecell[c]{#2}}}}

\newcommand{\mc}{\multicolumn}


\title{tablemaker Documentation}
\date{October 18, 2013}
\release{1.00}
\author{Jake C. Torcasso}
\newcommand{\sphinxlogo}{}
\renewcommand{\releasename}{Release}
\makeindex

\makeatletter
\def\PYG@reset{\let\PYG@it=\relax \let\PYG@bf=\relax%
    \let\PYG@ul=\relax \let\PYG@tc=\relax%
    \let\PYG@bc=\relax \let\PYG@ff=\relax}
\def\PYG@tok#1{\csname PYG@tok@#1\endcsname}
\def\PYG@toks#1+{\ifx\relax#1\empty\else%
    \PYG@tok{#1}\expandafter\PYG@toks\fi}
\def\PYG@do#1{\PYG@bc{\PYG@tc{\PYG@ul{%
    \PYG@it{\PYG@bf{\PYG@ff{#1}}}}}}}
\def\PYG#1#2{\PYG@reset\PYG@toks#1+\relax+\PYG@do{#2}}

\expandafter\def\csname PYG@tok@gd\endcsname{\def\PYG@tc##1{\textcolor[rgb]{0.63,0.00,0.00}{##1}}}
\expandafter\def\csname PYG@tok@gu\endcsname{\let\PYG@bf=\textbf\def\PYG@tc##1{\textcolor[rgb]{0.50,0.00,0.50}{##1}}}
\expandafter\def\csname PYG@tok@gt\endcsname{\def\PYG@tc##1{\textcolor[rgb]{0.00,0.27,0.87}{##1}}}
\expandafter\def\csname PYG@tok@gs\endcsname{\let\PYG@bf=\textbf}
\expandafter\def\csname PYG@tok@gr\endcsname{\def\PYG@tc##1{\textcolor[rgb]{1.00,0.00,0.00}{##1}}}
\expandafter\def\csname PYG@tok@cm\endcsname{\let\PYG@it=\textit\def\PYG@tc##1{\textcolor[rgb]{0.25,0.50,0.56}{##1}}}
\expandafter\def\csname PYG@tok@vg\endcsname{\def\PYG@tc##1{\textcolor[rgb]{0.73,0.38,0.84}{##1}}}
\expandafter\def\csname PYG@tok@m\endcsname{\def\PYG@tc##1{\textcolor[rgb]{0.13,0.50,0.31}{##1}}}
\expandafter\def\csname PYG@tok@mh\endcsname{\def\PYG@tc##1{\textcolor[rgb]{0.13,0.50,0.31}{##1}}}
\expandafter\def\csname PYG@tok@cs\endcsname{\def\PYG@tc##1{\textcolor[rgb]{0.25,0.50,0.56}{##1}}\def\PYG@bc##1{\setlength{\fboxsep}{0pt}\colorbox[rgb]{1.00,0.94,0.94}{\strut ##1}}}
\expandafter\def\csname PYG@tok@ge\endcsname{\let\PYG@it=\textit}
\expandafter\def\csname PYG@tok@vc\endcsname{\def\PYG@tc##1{\textcolor[rgb]{0.73,0.38,0.84}{##1}}}
\expandafter\def\csname PYG@tok@il\endcsname{\def\PYG@tc##1{\textcolor[rgb]{0.13,0.50,0.31}{##1}}}
\expandafter\def\csname PYG@tok@go\endcsname{\def\PYG@tc##1{\textcolor[rgb]{0.20,0.20,0.20}{##1}}}
\expandafter\def\csname PYG@tok@cp\endcsname{\def\PYG@tc##1{\textcolor[rgb]{0.00,0.44,0.13}{##1}}}
\expandafter\def\csname PYG@tok@gi\endcsname{\def\PYG@tc##1{\textcolor[rgb]{0.00,0.63,0.00}{##1}}}
\expandafter\def\csname PYG@tok@gh\endcsname{\let\PYG@bf=\textbf\def\PYG@tc##1{\textcolor[rgb]{0.00,0.00,0.50}{##1}}}
\expandafter\def\csname PYG@tok@ni\endcsname{\let\PYG@bf=\textbf\def\PYG@tc##1{\textcolor[rgb]{0.84,0.33,0.22}{##1}}}
\expandafter\def\csname PYG@tok@nl\endcsname{\let\PYG@bf=\textbf\def\PYG@tc##1{\textcolor[rgb]{0.00,0.13,0.44}{##1}}}
\expandafter\def\csname PYG@tok@nn\endcsname{\let\PYG@bf=\textbf\def\PYG@tc##1{\textcolor[rgb]{0.05,0.52,0.71}{##1}}}
\expandafter\def\csname PYG@tok@no\endcsname{\def\PYG@tc##1{\textcolor[rgb]{0.38,0.68,0.84}{##1}}}
\expandafter\def\csname PYG@tok@na\endcsname{\def\PYG@tc##1{\textcolor[rgb]{0.25,0.44,0.63}{##1}}}
\expandafter\def\csname PYG@tok@nb\endcsname{\def\PYG@tc##1{\textcolor[rgb]{0.00,0.44,0.13}{##1}}}
\expandafter\def\csname PYG@tok@nc\endcsname{\let\PYG@bf=\textbf\def\PYG@tc##1{\textcolor[rgb]{0.05,0.52,0.71}{##1}}}
\expandafter\def\csname PYG@tok@nd\endcsname{\let\PYG@bf=\textbf\def\PYG@tc##1{\textcolor[rgb]{0.33,0.33,0.33}{##1}}}
\expandafter\def\csname PYG@tok@ne\endcsname{\def\PYG@tc##1{\textcolor[rgb]{0.00,0.44,0.13}{##1}}}
\expandafter\def\csname PYG@tok@nf\endcsname{\def\PYG@tc##1{\textcolor[rgb]{0.02,0.16,0.49}{##1}}}
\expandafter\def\csname PYG@tok@si\endcsname{\let\PYG@it=\textit\def\PYG@tc##1{\textcolor[rgb]{0.44,0.63,0.82}{##1}}}
\expandafter\def\csname PYG@tok@s2\endcsname{\def\PYG@tc##1{\textcolor[rgb]{0.25,0.44,0.63}{##1}}}
\expandafter\def\csname PYG@tok@vi\endcsname{\def\PYG@tc##1{\textcolor[rgb]{0.73,0.38,0.84}{##1}}}
\expandafter\def\csname PYG@tok@nt\endcsname{\let\PYG@bf=\textbf\def\PYG@tc##1{\textcolor[rgb]{0.02,0.16,0.45}{##1}}}
\expandafter\def\csname PYG@tok@nv\endcsname{\def\PYG@tc##1{\textcolor[rgb]{0.73,0.38,0.84}{##1}}}
\expandafter\def\csname PYG@tok@s1\endcsname{\def\PYG@tc##1{\textcolor[rgb]{0.25,0.44,0.63}{##1}}}
\expandafter\def\csname PYG@tok@gp\endcsname{\let\PYG@bf=\textbf\def\PYG@tc##1{\textcolor[rgb]{0.78,0.36,0.04}{##1}}}
\expandafter\def\csname PYG@tok@sh\endcsname{\def\PYG@tc##1{\textcolor[rgb]{0.25,0.44,0.63}{##1}}}
\expandafter\def\csname PYG@tok@ow\endcsname{\let\PYG@bf=\textbf\def\PYG@tc##1{\textcolor[rgb]{0.00,0.44,0.13}{##1}}}
\expandafter\def\csname PYG@tok@sx\endcsname{\def\PYG@tc##1{\textcolor[rgb]{0.78,0.36,0.04}{##1}}}
\expandafter\def\csname PYG@tok@bp\endcsname{\def\PYG@tc##1{\textcolor[rgb]{0.00,0.44,0.13}{##1}}}
\expandafter\def\csname PYG@tok@c1\endcsname{\let\PYG@it=\textit\def\PYG@tc##1{\textcolor[rgb]{0.25,0.50,0.56}{##1}}}
\expandafter\def\csname PYG@tok@kc\endcsname{\let\PYG@bf=\textbf\def\PYG@tc##1{\textcolor[rgb]{0.00,0.44,0.13}{##1}}}
\expandafter\def\csname PYG@tok@c\endcsname{\let\PYG@it=\textit\def\PYG@tc##1{\textcolor[rgb]{0.25,0.50,0.56}{##1}}}
\expandafter\def\csname PYG@tok@mf\endcsname{\def\PYG@tc##1{\textcolor[rgb]{0.13,0.50,0.31}{##1}}}
\expandafter\def\csname PYG@tok@err\endcsname{\def\PYG@bc##1{\setlength{\fboxsep}{0pt}\fcolorbox[rgb]{1.00,0.00,0.00}{1,1,1}{\strut ##1}}}
\expandafter\def\csname PYG@tok@kd\endcsname{\let\PYG@bf=\textbf\def\PYG@tc##1{\textcolor[rgb]{0.00,0.44,0.13}{##1}}}
\expandafter\def\csname PYG@tok@ss\endcsname{\def\PYG@tc##1{\textcolor[rgb]{0.32,0.47,0.09}{##1}}}
\expandafter\def\csname PYG@tok@sr\endcsname{\def\PYG@tc##1{\textcolor[rgb]{0.14,0.33,0.53}{##1}}}
\expandafter\def\csname PYG@tok@mo\endcsname{\def\PYG@tc##1{\textcolor[rgb]{0.13,0.50,0.31}{##1}}}
\expandafter\def\csname PYG@tok@mi\endcsname{\def\PYG@tc##1{\textcolor[rgb]{0.13,0.50,0.31}{##1}}}
\expandafter\def\csname PYG@tok@kn\endcsname{\let\PYG@bf=\textbf\def\PYG@tc##1{\textcolor[rgb]{0.00,0.44,0.13}{##1}}}
\expandafter\def\csname PYG@tok@o\endcsname{\def\PYG@tc##1{\textcolor[rgb]{0.40,0.40,0.40}{##1}}}
\expandafter\def\csname PYG@tok@kr\endcsname{\let\PYG@bf=\textbf\def\PYG@tc##1{\textcolor[rgb]{0.00,0.44,0.13}{##1}}}
\expandafter\def\csname PYG@tok@s\endcsname{\def\PYG@tc##1{\textcolor[rgb]{0.25,0.44,0.63}{##1}}}
\expandafter\def\csname PYG@tok@kp\endcsname{\def\PYG@tc##1{\textcolor[rgb]{0.00,0.44,0.13}{##1}}}
\expandafter\def\csname PYG@tok@w\endcsname{\def\PYG@tc##1{\textcolor[rgb]{0.73,0.73,0.73}{##1}}}
\expandafter\def\csname PYG@tok@kt\endcsname{\def\PYG@tc##1{\textcolor[rgb]{0.56,0.13,0.00}{##1}}}
\expandafter\def\csname PYG@tok@sc\endcsname{\def\PYG@tc##1{\textcolor[rgb]{0.25,0.44,0.63}{##1}}}
\expandafter\def\csname PYG@tok@sb\endcsname{\def\PYG@tc##1{\textcolor[rgb]{0.25,0.44,0.63}{##1}}}
\expandafter\def\csname PYG@tok@k\endcsname{\let\PYG@bf=\textbf\def\PYG@tc##1{\textcolor[rgb]{0.00,0.44,0.13}{##1}}}
\expandafter\def\csname PYG@tok@se\endcsname{\let\PYG@bf=\textbf\def\PYG@tc##1{\textcolor[rgb]{0.25,0.44,0.63}{##1}}}
\expandafter\def\csname PYG@tok@sd\endcsname{\let\PYG@it=\textit\def\PYG@tc##1{\textcolor[rgb]{0.25,0.44,0.63}{##1}}}

\def\PYGZbs{\char`\\}
\def\PYGZus{\char`\_}
\def\PYGZob{\char`\{}
\def\PYGZcb{\char`\}}
\def\PYGZca{\char`\^}
\def\PYGZam{\char`\&}
\def\PYGZlt{\char`\<}
\def\PYGZgt{\char`\>}
\def\PYGZsh{\char`\#}
\def\PYGZpc{\char`\%}
\def\PYGZdl{\char`\$}
\def\PYGZhy{\char`\-}
\def\PYGZsq{\char`\'}
\def\PYGZdq{\char`\"}
\def\PYGZti{\char`\~}
% for compatibility with earlier versions
\def\PYGZat{@}
\def\PYGZlb{[}
\def\PYGZrb{]}
\makeatother

\begin{document}

\maketitle
\tableofcontents
\phantomsection\label{index::doc}\phantomsection\label{summary:module-tablemaker}\index{tablemaker (module)}


Controls to take data and format into a LateX table. These controls 
include some functions, and two classes.

Classes:
\begin{itemize}
\item {} 
\emph{Row}: a row to go inside a table

\item {} 
\emph{Table}: a table to be formatted

\end{itemize}

Functions:
\begin{itemize}
\item {} 
\emph{texCommands()}: return a string with the required LateX header

\item {} \begin{description}
\item[{\emph{read()}: \textbar{} returns a dictionary of sheets, the values of which can}] \leavevmode
be used to instantiate a \emph{Table} object

\end{description}

\end{itemize}


\chapter{How to Use this module}
\label{summary::doc}\label{summary:how-to-use-this-module}\label{summary:summary}\begin{enumerate}
\item {} 
Ensure the module is in your path

\item {} 
Import the module with \code{import tablemaker as tm}

\item {} 
\begin{DUlineblock}{0em}
\item[] Create a dictionary with row numbers (0,1,2,..) as keys and 
lists containing values for that row
\end{DUlineblock}
\begin{enumerate}
\item {} 
create this manually

\item {} 
Use a .xls file and the \emph{read()} function

\end{enumerate}

\item {} 
\begin{DUlineblock}{0em}
\item[] After instantiating a \emph{Table} object, you can format the table 
using the \emph{Table} methods or you can access \emph{Row} methods by 
calling out the rows in \emph{Table}.rows
\end{DUlineblock}

\item {} 
\begin{DUlineblock}{0em}
\item[] Create a tex representation of the table as a string using the 
\emph{Table.makeTex()} method and write to file. 
\end{DUlineblock}

\end{enumerate}


\section{Example}
\label{summary:example}
\begin{Verbatim}[commandchars=\\\{\}]
\PYG{k+kn}{import} \PYG{n+nn}{tablemaker} \PYG{k+kn}{as} \PYG{n+nn}{tm}

\PYG{n}{rowdict} \PYG{o}{=}  \PYG{p}{\PYGZob{}}\PYG{l+m+mi}{0}\PYG{p}{:}\PYG{p}{[}\PYG{l+s}{\PYGZsq{}}\PYG{l+s}{Year}\PYG{l+s}{\PYGZsq{}}\PYG{p}{,} \PYG{l+s}{\PYGZsq{}}\PYG{l+s}{Expenses}\PYG{l+s}{\PYGZsq{}}\PYG{p}{]}\PYG{p}{,}
                        \PYG{l+m+mi}{1}\PYG{p}{:}\PYG{p}{[}\PYG{l+m+mi}{2004}\PYG{p}{,}   \PYG{l+m+mi}{3495}\PYG{p}{]}\PYG{p}{,}
                        \PYG{l+m+mi}{2}\PYG{p}{:}\PYG{p}{[}\PYG{l+m+mi}{2005}\PYG{p}{,}   \PYG{l+m+mi}{4500}\PYG{p}{]}\PYG{p}{\PYGZcb{}}

\PYG{n}{table} \PYG{o}{=} \PYG{n}{tm}\PYG{o}{.}\PYG{n}{Table}\PYG{p}{(}\PYG{n}{rowdict}\PYG{p}{)}

\PYG{n}{table}\PYG{o}{.}\PYG{n}{setTitle}\PYG{p}{(}\PYG{l+s}{\PYGZsq{}}\PYG{l+s}{My Yearly Expenses}\PYG{l+s}{\PYGZsq{}}\PYG{p}{)}

\PYG{n}{rows} \PYG{o}{=} \PYG{n}{table}\PYG{o}{.}\PYG{n}{rows}

\PYG{n}{rows}\PYG{p}{[}\PYG{l+m+mi}{0}\PYG{p}{]}\PYG{o}{.}\PYG{n}{align}\PYG{p}{(}\PYG{p}{[}\PYG{l+s}{\PYGZsq{}}\PYG{l+s}{c}\PYG{l+s}{\PYGZsq{}}\PYG{p}{,} \PYG{l+s}{\PYGZsq{}}\PYG{l+s}{c}\PYG{l+s}{\PYGZsq{}}\PYG{p}{]}\PYG{p}{)}

\PYG{n}{texfile} \PYG{o}{=} \PYG{n+nb}{open}\PYG{p}{(}\PYG{l+s}{\PYGZsq{}}\PYG{l+s}{myExpenses.tex}\PYG{l+s}{\PYGZsq{}}\PYG{p}{,} \PYG{l+s}{\PYGZsq{}}\PYG{l+s}{w}\PYG{l+s}{\PYGZsq{}}\PYG{p}{)}

\PYG{n}{texfile}\PYG{o}{.}\PYG{n}{write}\PYG{p}{(}\PYG{n}{table}\PYG{o}{.}\PYG{n}{makeTex}\PYG{p}{(}\PYG{p}{)}\PYG{p}{)}

\PYG{n}{texfile}\PYG{o}{.}\PYG{n}{close}\PYG{p}{(}\PYG{p}{)}
\end{Verbatim}


\chapter{Dependences}
\label{summary:dependences}
Python Packages:
\begin{itemize}
\item {} 
xlrd

\item {} 
numpy

\end{itemize}

LateX Packages:
\begin{itemize}
\item {} 
booktabs

\item {} 
makecell

\item {} 
graphicx

\item {} 
fontenc

\item {} 
longtable

\item {} 
rotating

\item {} 
multirow

\end{itemize}

LateX Commands:

\begin{Verbatim}[commandchars=\\\{\}]
- \PYGZbs{}renewcommand\PYGZob{}\PYGZbs{}rothead\PYGZcb{}[2][60]\PYGZob{}\PYGZbs{}makebox[9mm][c]\PYGZob{}\PYGZbs{}rotatebox\PYGZob{}\#1\PYGZcb{}\PYGZob{}\PYGZbs{}makecell[c]\PYGZob{}\#2\PYGZcb{}\PYGZcb{}\PYGZcb{}\PYGZcb{}
- \PYGZbs{}newcommand\PYGZob{}\PYGZbs{}mc\PYGZcb{}\PYGZob{}\PYGZbs{}multicolumn\PYGZcb{}
\end{Verbatim}


\chapter{Module Objects}
\label{summary:module-objects}

\section{Functions}
\label{summary:functions}\index{read() (in module tablemaker)}

\begin{fulllineitems}
\phantomsection\label{summary:tablemaker.read}\pysiglinewithargsret{\code{tablemaker.}\bfcode{read}}{\emph{file\_}}{}
Creates a dictionary of sheets from \emph{file\_}, an .xls file, 
with values as dictionaries, matching row numbers to lists of 
elements in the row.

\end{fulllineitems}

\index{texCommands() (in module tablemaker)}

\begin{fulllineitems}
\phantomsection\label{summary:tablemaker.texCommands}\pysiglinewithargsret{\code{tablemaker.}\bfcode{texCommands}}{}{}
Returns a string to go in the preamble of your LaTeX
document. It includes required packages and commands, 
used to compile in LateX.

\end{fulllineitems}



\section{Classes}
\label{summary:classes}\index{Table (class in tablemaker)}

\begin{fulllineitems}
\phantomsection\label{summary:tablemaker.Table}\pysiglinewithargsret{\strong{class }\code{tablemaker.}\bfcode{Table}}{\emph{result\_dict}, \emph{type\_='table'}, \emph{fillempty=None}, \emph{merge=False}}{}~\index{makeTex() (tablemaker.Table method)}

\begin{fulllineitems}
\phantomsection\label{summary:tablemaker.Table.makeTex}\pysiglinewithargsret{\bfcode{makeTex}}{}{}
Generates TeX string for a table.

You may manipulate \emph{self.rows} or the Table prior to using this
function to meet your specific needs.

\end{fulllineitems}

\index{setFloat() (tablemaker.Table method)}

\begin{fulllineitems}
\phantomsection\label{summary:tablemaker.Table.setFloat}\pysiglinewithargsret{\bfcode{setFloat}}{\emph{floatChar='h'}}{}
Sets \emph{floatChar} to specified value, default is `h'

\end{fulllineitems}

\index{setFontSize() (tablemaker.Table method)}

\begin{fulllineitems}
\phantomsection\label{summary:tablemaker.Table.setFontSize}\pysiglinewithargsret{\bfcode{setFontSize}}{\emph{fontsize}}{}
\emph{fontsize}: a string in {[}'tiny', `scriptsize', `footnotesize', `small', `normalsize',
`large', `Large', `LARGE', `huge', `HUGE'{]}

\end{fulllineitems}

\index{setHlines() (tablemaker.Table method)}

\begin{fulllineitems}
\phantomsection\label{summary:tablemaker.Table.setHlines}\pysiglinewithargsret{\bfcode{setHlines}}{\emph{tupleofints=(1}, \emph{1)}}{}
Sets the number of horizontal lines to put at the 
top and bottom of the table.

Default is (1,1) \textasciitilde{} (beginning, end)

\end{fulllineitems}

\index{setLabel() (tablemaker.Table method)}

\begin{fulllineitems}
\phantomsection\label{summary:tablemaker.Table.setLabel}\pysiglinewithargsret{\bfcode{setLabel}}{\emph{label}}{}
\emph{label}: a string to put as a label for the table

\end{fulllineitems}

\index{setNotes() (tablemaker.Table method)}

\begin{fulllineitems}
\phantomsection\label{summary:tablemaker.Table.setNotes}\pysiglinewithargsret{\bfcode{setNotes}}{\emph{notes}}{}
\emph{notes}: footnotes to put into table (a string in TeX format)

\end{fulllineitems}

\index{setRepeat() (tablemaker.Table method)}

\begin{fulllineitems}
\phantomsection\label{summary:tablemaker.Table.setRepeat}\pysiglinewithargsret{\bfcode{setRepeat}}{\emph{rows}}{}
Used only if \emph{self.type\_} == `longtable'
Specifies \emph{rows} to repeat at the top of every page. 
The first row is indexed at 0.

\end{fulllineitems}

\index{setRows() (tablemaker.Table method)}

\begin{fulllineitems}
\phantomsection\label{summary:tablemaker.Table.setRows}\pysiglinewithargsret{\bfcode{setRows}}{}{}
Creates a list of rowobjects and sets \emph{self.rows} to this list

\end{fulllineitems}

\index{setTableAlignment() (tablemaker.Table method)}

\begin{fulllineitems}
\phantomsection\label{summary:tablemaker.Table.setTableAlignment}\pysiglinewithargsret{\bfcode{setTableAlignment}}{\emph{alignString}}{}
Set the \emph{alignString} to a string which will go into
the argument of the tabular environment in LateX.

Ex: cccccccc

\end{fulllineitems}

\index{setTitle() (tablemaker.Table method)}

\begin{fulllineitems}
\phantomsection\label{summary:tablemaker.Table.setTitle}\pysiglinewithargsret{\bfcode{setTitle}}{\emph{title}}{}
\emph{title}: a string to put into the caption of the table

\end{fulllineitems}

\index{setType() (tablemaker.Table method)}

\begin{fulllineitems}
\phantomsection\label{summary:tablemaker.Table.setType}\pysiglinewithargsret{\bfcode{setType}}{\emph{type\_}}{}
Set the type of the table. Where \emph{type\_} is in 
{[}'table', `sidewaystable', `longtable'{]}

\end{fulllineitems}


\end{fulllineitems}

\index{Row (class in tablemaker)}

\begin{fulllineitems}
\phantomsection\label{summary:tablemaker.Row}\pysiglinewithargsret{\strong{class }\code{tablemaker.}\bfcode{Row}}{\emph{rowvalues}, \emph{fillempty=None}, \emph{merge=False}}{}~\index{addhlines() (tablemaker.Row method)}

\begin{fulllineitems}
\phantomsection\label{summary:tablemaker.Row.addhlines}\pysiglinewithargsret{\bfcode{addhlines}}{\emph{listoftuples=None}, \emph{title=False}}{}
\emph{listoftuples} like {[}(1,2), (5,6),...{]}
by default \emph{listoftuples} is None and function
sets self.hlines to `hline'

numbers in tuples mean where to stop and
start an underlying horizontal line

1 is the first column in the row

Sets \emph{self.hlines} to a string LateX can
interpret as some sort of horizontal line

\end{fulllineitems}

\index{align() (tablemaker.Row method)}

\begin{fulllineitems}
\phantomsection\label{summary:tablemaker.Row.align}\pysiglinewithargsret{\bfcode{align}}{\emph{alignlist}}{}
Sets row alignment based on LateX convention:
{\color{red}\bfseries{}\textbar{}c\textbar{}}, {\color{red}\bfseries{}\textbar{}c, c\textbar{}}, c, p\{2cm\}, None (and others)

\emph{alignlist} should be of length equal to number of
columns in row. If \emph{self.merge} == True, then \emph{alignlist}
same length as nonempty values in row.
\begin{quote}

Example: a row with one empty column and four nonempty
\begin{description}
\item[{if \emph{self.merge} == True:}] \leavevmode
\emph{alignlist} = {[}'c', `{\color{red}\bfseries{}\textbar{}c', `p\{1cm\}', `\textbar{}l\textbar{}}`{]}

\item[{else:}] \leavevmode
\#add another element to the list for the empty cell
\emph{alignlist} = {[}'c', `c', `{\color{red}\bfseries{}\textbar{}c', `p\{1cm\}', `\textbar{}l\textbar{}}`{]}

\end{description}

The order reflects the order of the columns in the row
\end{quote}

\end{fulllineitems}

\index{autoFormat() (tablemaker.Row method)}

\begin{fulllineitems}
\phantomsection\label{summary:tablemaker.Row.autoFormat}\pysiglinewithargsret{\bfcode{autoFormat}}{\emph{auto=True}}{}
If \emph{auto} is True, will use built-in formatter, \emph{self.formatValue()}, 
to format the elements in the list

\end{fulllineitems}

\index{buildHline() (tablemaker.Row method)}

\begin{fulllineitems}
\phantomsection\label{summary:tablemaker.Row.buildHline}\pysiglinewithargsret{\bfcode{buildHline}}{}{}
Invoked by \emph{self.addhlines()} if parameter title = True
\begin{description}
\item[{Returns list of tuples (ex. {[}(1,2), (3,4){]}}] \leavevmode
in the above example, horizontal lines would
span columns 1 to 2 and then another from columns 3 to 4

\end{description}

\end{fulllineitems}

\index{formatValue() (tablemaker.Row method)}

\begin{fulllineitems}
\phantomsection\label{summary:tablemaker.Row.formatValue}\pysiglinewithargsret{\bfcode{formatValue}}{\emph{val}}{}
Formats \emph{val} (a value from a table cell)

\end{fulllineitems}

\index{getValues() (tablemaker.Row method)}

\begin{fulllineitems}
\phantomsection\label{summary:tablemaker.Row.getValues}\pysiglinewithargsret{\bfcode{getValues}}{}{}
returns dict of row values and their column number,
starting with column 0

if \emph{self.merge} == True, will ignore rows
that are empty strings

\end{fulllineitems}

\index{makeAlignment() (tablemaker.Row method)}

\begin{fulllineitems}
\phantomsection\label{summary:tablemaker.Row.makeAlignment}\pysiglinewithargsret{\bfcode{makeAlignment}}{}{}
Generates the alignment for a row. Alignment is of
length equal to the number of non-empty values in row.

\end{fulllineitems}

\index{makeTex() (tablemaker.Row method)}

\begin{fulllineitems}
\phantomsection\label{summary:tablemaker.Row.makeTex}\pysiglinewithargsret{\bfcode{makeTex}}{}{}
Generates TeX string for a table row. Empty columns in row
are automatically turned into a multicolumn with the nearest
non-empty column on the left.

\end{fulllineitems}

\index{mergeEmpty() (tablemaker.Row method)}

\begin{fulllineitems}
\phantomsection\label{summary:tablemaker.Row.mergeEmpty}\pysiglinewithargsret{\bfcode{mergeEmpty}}{}{}
Merges empty cells to closest non-empty cell to the left.

\end{fulllineitems}

\index{rotateContents() (tablemaker.Row method)}

\begin{fulllineitems}
\phantomsection\label{summary:tablemaker.Row.rotateContents}\pysiglinewithargsret{\bfcode{rotateContents}}{}{}
Sets \emph{self.rotate} to True

\end{fulllineitems}

\index{setBottomSpacing() (tablemaker.Row method)}

\begin{fulllineitems}
\phantomsection\label{summary:tablemaker.Row.setBottomSpacing}\pysiglinewithargsret{\bfcode{setBottomSpacing}}{\emph{row\_space=`0.25cm'}}{}
Sets the amount of padding below
the row

\emph{row\_space} must specify a length in LateX
\begin{description}
\item[{example:    1cm}] \leavevmode
or
0.25cm
or 
25mm
or
0.25

\end{description}

\end{fulllineitems}

\index{setFormat() (tablemaker.Row method)}

\begin{fulllineitems}
\phantomsection\label{summary:tablemaker.Row.setFormat}\pysiglinewithargsret{\bfcode{setFormat}}{\emph{texcommand}}{}
if \emph{texcommand} is a string
Will wrap the contents of each non-empty cell in the texcommand.

Ex. \emph{texcommand} = `textbf'

if \emph{texcommand} is a list of strings
Will wrap the contents of each non-empty cell according to the 
command with the same index in the list of strings
\begin{description}
\item[{Ex. row = {[}'Hello', `you'{]}}] \leavevmode
texcommand = {[}'textbf', `'{]}

\end{description}

\end{fulllineitems}

\index{setTopSpacing() (tablemaker.Row method)}

\begin{fulllineitems}
\phantomsection\label{summary:tablemaker.Row.setTopSpacing}\pysiglinewithargsret{\bfcode{setTopSpacing}}{\emph{row\_space=`0.25cm'}}{}
Sets the amount of padding below
the row

\emph{row\_space} must specify a length in LateX
\begin{description}
\item[{example:    1cm}] \leavevmode
or
0.25cm
or 
25mm
or
0.25

\end{description}

\end{fulllineitems}


\end{fulllineitems}



\chapter{Detailed Tutorial}
\label{tutorial:detailed-tutorial}\label{tutorial::doc}
First let's save the current directory because our example .xls table file
is located here. If you have downloaded this material on your local machine
ensure the directory you save to tablepath contains testTable.xls.

\begin{Verbatim}[commandchars=\\\{\}]
\PYG{c}{\PYGZsh{}Importing standard libraries}
\PYG{k+kn}{import} \PYG{n+nn}{os}
\PYG{k+kn}{import} \PYG{n+nn}{sys}

\PYG{n}{tablepath} \PYG{o}{=} \PYG{n}{os}\PYG{o}{.}\PYG{n}{getcwd}\PYG{p}{(}\PYG{p}{)}
\end{Verbatim}

Let's start by importing tablemaker.py. If you have the environment variables
`erc' setup with `ercTools' as a subdirectory and tablemaker.py in that
directory, then use the following code:

\begin{Verbatim}[commandchars=\\\{\}]
\PYG{c}{\PYGZsh{}Extending Path to directory of tablemaker module}
\PYG{n}{os}\PYG{o}{.}\PYG{n}{chdir}\PYG{p}{(}\PYG{n}{os}\PYG{o}{.}\PYG{n}{environ}\PYG{p}{[}\PYG{l+s}{\PYGZsq{}}\PYG{l+s}{erc}\PYG{l+s}{\PYGZsq{}}\PYG{p}{]}\PYG{p}{)}\PYG{p}{;} \PYG{n}{os}\PYG{o}{.}\PYG{n}{chdir}\PYG{p}{(}\PYG{l+s}{\PYGZdq{}}\PYG{l+s}{ercTools}\PYG{l+s}{\PYGZdq{}}\PYG{p}{)}\PYG{p}{;} \PYG{n}{toolspath} \PYG{o}{=} \PYG{n}{os}\PYG{o}{.}\PYG{n}{getcwd}\PYG{p}{(}\PYG{p}{)}
\PYG{n}{sys}\PYG{o}{.}\PYG{n}{path}\PYG{o}{.}\PYG{n}{append}\PYG{p}{(}\PYG{n}{toolspath}\PYG{p}{)}
\end{Verbatim}

We have added the path to ercTools temporarily to the system path. The system
path contains directories Python will use to search for modules. Thus, by adding
ercTools to this path, Python will be able to find tablemaker.py. So let's
import the module.

\begin{Verbatim}[commandchars=\\\{\}]
\PYG{k+kn}{import} \PYG{n+nn}{tablemaker} \PYG{k+kn}{as} \PYG{n+nn}{tm}
\end{Verbatim}

To create a latex table using tablemaker we can use two methods:
\begin{enumerate}
\item {} 
use existing xls file and create tables from this file

\item {} 
create a dictionary of lists which contain the elements of a row

\end{enumerate}

We will demonstrate the first method here

\begin{Verbatim}[commandchars=\\\{\}]
\PYG{c}{\PYGZsh{}change directory to tablepath}
\PYG{n}{os}\PYG{o}{.}\PYG{n}{chdir}\PYG{p}{(}\PYG{n}{tablepath}\PYG{p}{)}

\PYG{c}{\PYGZsh{}First, let\PYGZsq{}s draw in the data}
\PYG{n}{sheetDict} \PYG{o}{=} \PYG{n}{tm}\PYG{o}{.}\PYG{n}{read}\PYG{p}{(}\PYG{l+s}{\PYGZsq{}}\PYG{l+s}{testTable.xls}\PYG{l+s}{\PYGZsq{}}\PYG{p}{)}
\end{Verbatim}

When we instantiate a table object from tablemaker we need to tell it
which sheet of data to use. `testTable.xls' contains `Sheet1' and `Sheet2'.
We will begin with `Sheet1'

\begin{Verbatim}[commandchars=\\\{\}]
\PYG{n}{table} \PYG{o}{=} \PYG{n}{tm}\PYG{o}{.}\PYG{n}{Table}\PYG{p}{(}\PYG{n}{sheetDict}\PYG{p}{[}\PYG{l+s}{\PYGZsq{}}\PYG{l+s}{Sheet1}\PYG{l+s}{\PYGZsq{}}\PYG{p}{]}\PYG{p}{)}
\end{Verbatim}

We gave the basic arguments to create a table, but we could have specified
more arguments in the call. For example:

\begin{Verbatim}[commandchars=\\\{\}]
\PYG{n}{tm}\PYG{o}{.}\PYG{n}{Table}\PYG{p}{(}\PYG{n}{result\PYGZus{}dict}\PYG{p}{,} \PYG{n}{type\PYGZus{}} \PYG{o}{=} \PYG{l+s}{\PYGZsq{}}\PYG{l+s}{table}\PYG{l+s}{\PYGZsq{}}\PYG{p}{,} \PYG{n}{fillempty} \PYG{o}{=} \PYG{n+nb+bp}{None}\PYG{p}{,} \PYG{n}{merge} \PYG{o}{=} \PYG{n+nb+bp}{False}\PYG{p}{)}
\end{Verbatim}

Now we can write the tex of the table without further modifications. And we
can see how it looks. We call the makeTex() of the table object to produce
a string to insert into the tex file.

\begin{Verbatim}[commandchars=\\\{\}]
\PYG{n}{texfile} \PYG{o}{=} \PYG{n+nb}{open}\PYG{p}{(}\PYG{l+s}{\PYGZsq{}}\PYG{l+s}{test1.tex}\PYG{l+s}{\PYGZsq{}}\PYG{p}{,} \PYG{l+s}{\PYGZsq{}}\PYG{l+s}{w}\PYG{l+s}{\PYGZsq{}}\PYG{p}{)}
\PYG{n}{texfile}\PYG{o}{.}\PYG{n}{write}\PYG{p}{(}\PYG{n}{table}\PYG{o}{.}\PYG{n}{makeTex}\PYG{p}{(}\PYG{p}{)}\PYG{p}{)}
\PYG{n}{texfile}\PYG{o}{.}\PYG{n}{close}\PYG{p}{(}\PYG{p}{)}
\end{Verbatim}

Here is a first look at the table.
%Created Using: /home/jake/Documents/repos/erc/ercTools/tablemaker.py
\begin{table}[h]
\begin{center} \small
\begin{tabular}{ccccccccccccc}
\hline 

 & Head Start Group &  &  &  &  &  & Control Group &  &  &  &  &  \\

 & Treated &  &  & No shows &  &  & Control &  &  & Crossover &  &  \\

Measure & obs & mean & std & obs & mean & std & obs & mean & std & obs & mean & std \\

Parent Measures &  &  &  &  &  &  &  &  &  &  &  &  \\

Mother's education & 99 & 10.60 & 4.71 & 34 & 6.27 & 4.42 & 33 & 6.20 & 4.41 & 34 & 6.27 & 4.42 \\

Father's education & 100 & 10.67 & 4.71 & 35 & 6.33 & 4.42 & 34 & 6.27 & 4.42 & 35 & 6.33 & 4.42 \\

Combined education & 101 & 10.73 & 4.72 & 36 & 6.40 & 4.43 & 35 & 6.33 & 4.42 & 36 & 6.40 & 4.43 \\

Mother's occupation level & 102 & 10.80 & 4.72 & 37 & 6.47 & 4.43 & 36 & 6.40 & 4.43 & 37 & 6.47 & 4.43 \\

Father's occupation level & 103 & 10.87 & 4.72 & 38 & 6.53 & 4.44 & 37 & 6.47 & 4.43 & 38 & 6.53 & 4.44 \\

Child Measures &  &  &  &  &  &  &  &  &  &  &  &  \\

IQ Score & 99 & 10.60 & 4.71 & 34 & 6.27 & 4.42 & 56 & 7.73 & 4.52 & 56 & 7.73 & 4.52 \\

Behavior Index & 100 & 10.67 & 4.71 & 35 & 6.33 & 4.42 & 57 & 7.80 & 4.52 & 57 & 7.80 & 4.52 \\

PPVT Score & 101 & 10.73 & 4.72 & 36 & 6.40 & 4.43 & 58 & 7.87 & 4.52 & 58 & 7.87 & 4.52 \\

Timeouts & 102 & 10.80 & 4.72 & 37 & 6.47 & 4.43 & 59 & 7.93 & 4.53 & 59 & 7.93 & 4.53 \\

\hline 
\end{tabular}
\end{center}
\end{table}
It does not appear the table even fits on the page. We can either turn the table
sideways or merge some cells. First, let's try turning sideways.

\begin{Verbatim}[commandchars=\\\{\}]
\PYG{n}{table}\PYG{o}{.}\PYG{n}{setType}\PYG{p}{(}\PYG{l+s}{\PYGZsq{}}\PYG{l+s}{sidewaystable}\PYG{l+s}{\PYGZsq{}}\PYG{p}{)}

\PYG{n}{texfile} \PYG{o}{=} \PYG{n+nb}{open}\PYG{p}{(}\PYG{l+s}{\PYGZsq{}}\PYG{l+s}{test2.tex}\PYG{l+s}{\PYGZsq{}}\PYG{p}{,} \PYG{l+s}{\PYGZsq{}}\PYG{l+s}{w}\PYG{l+s}{\PYGZsq{}}\PYG{p}{)}
\PYG{n}{texfile}\PYG{o}{.}\PYG{n}{write}\PYG{p}{(}\PYG{n}{table}\PYG{o}{.}\PYG{n}{makeTex}\PYG{p}{(}\PYG{p}{)}\PYG{p}{)}
\PYG{n}{texfile}\PYG{o}{.}\PYG{n}{close}\PYG{p}{(}\PYG{p}{)}
\end{Verbatim}

Here is what the sidewaystable looks like.
\clearpage
%Created Using: /home/jake/Documents/repos/erc/ercTools/tablemaker.py
\begin{sidewaystable}[h]
\begin{center} \small
\begin{tabular}{ccccccccccccc}
\hline 

 & Head Start Group &  &  &  &  &  & Control Group &  &  &  &  &  \\

 & Treated &  &  & No shows &  &  & Control &  &  & Crossover &  &  \\

Measure & obs & mean & std & obs & mean & std & obs & mean & std & obs & mean & std \\

Parent Measures &  &  &  &  &  &  &  &  &  &  &  &  \\

Mother's education & 99 & 10.60 & 4.71 & 34 & 6.27 & 4.42 & 33 & 6.20 & 4.41 & 34 & 6.27 & 4.42 \\

Father's education & 100 & 10.67 & 4.71 & 35 & 6.33 & 4.42 & 34 & 6.27 & 4.42 & 35 & 6.33 & 4.42 \\

Combined education & 101 & 10.73 & 4.72 & 36 & 6.40 & 4.43 & 35 & 6.33 & 4.42 & 36 & 6.40 & 4.43 \\

Mother's occupation level & 102 & 10.80 & 4.72 & 37 & 6.47 & 4.43 & 36 & 6.40 & 4.43 & 37 & 6.47 & 4.43 \\

Father's occupation level & 103 & 10.87 & 4.72 & 38 & 6.53 & 4.44 & 37 & 6.47 & 4.43 & 38 & 6.53 & 4.44 \\

Child Measures &  &  &  &  &  &  &  &  &  &  &  &  \\

IQ Score & 99 & 10.60 & 4.71 & 34 & 6.27 & 4.42 & 56 & 7.73 & 4.52 & 56 & 7.73 & 4.52 \\

Behavior Index & 100 & 10.67 & 4.71 & 35 & 6.33 & 4.42 & 57 & 7.80 & 4.52 & 57 & 7.80 & 4.52 \\

PPVT Score & 101 & 10.73 & 4.72 & 36 & 6.40 & 4.43 & 58 & 7.87 & 4.52 & 58 & 7.87 & 4.52 \\

Timeouts & 102 & 10.80 & 4.72 & 37 & 6.47 & 4.43 & 59 & 7.93 & 4.53 & 59 & 7.93 & 4.53 \\

\hline 
\end{tabular}
\end{center}
\end{sidewaystable}
\clearpage
We probably do not need the table sideways here, if we merge the right cells
all the data should fit.

\begin{Verbatim}[commandchars=\\\{\}]
\PYG{n}{table}\PYG{o}{.}\PYG{n}{setType}\PYG{p}{(}\PYG{l+s}{\PYGZsq{}}\PYG{l+s}{table}\PYG{l+s}{\PYGZsq{}}\PYG{p}{)}
\end{Verbatim}

In order to merge the appropriate rows, we will use the row by row
functionality of tablemaker. First we get a list of row objects from
the current table. The first row is indexed with 0. Then we can configure
the rows, one by one.

\begin{Verbatim}[commandchars=\\\{\}]
\PYG{n}{rows} \PYG{o}{=} \PYG{n}{table}\PYG{o}{.}\PYG{n}{rows}

\PYG{n}{rows}\PYG{p}{[}\PYG{l+m+mi}{0}\PYG{p}{]}\PYG{o}{.}\PYG{n}{mergeEmpty}\PYG{p}{(}\PYG{p}{)}                \PYG{c}{\PYGZsh{}merge the row, treat as two columns}
\PYG{n}{rows}\PYG{p}{[}\PYG{l+m+mi}{0}\PYG{p}{]}\PYG{o}{.}\PYG{n}{addhlines}\PYG{p}{(}\PYG{n}{title} \PYG{o}{=} \PYG{n+nb+bp}{True}\PYG{p}{)}     \PYG{c}{\PYGZsh{}title=True adds lines only under heading}
\PYG{n}{rows}\PYG{p}{[}\PYG{l+m+mi}{0}\PYG{p}{]}\PYG{o}{.}\PYG{n}{align}\PYG{p}{(}\PYG{p}{[}\PYG{l+s}{\PYGZsq{}}\PYG{l+s}{c}\PYG{l+s}{\PYGZsq{}}\PYG{p}{]}\PYG{o}{*}\PYG{l+m+mi}{2}\PYG{p}{)}              \PYG{c}{\PYGZsh{}centering the two columns}
\PYG{n}{rows}\PYG{p}{[}\PYG{l+m+mi}{0}\PYG{p}{]}\PYG{o}{.}\PYG{n}{setFormat}\PYG{p}{(}\PYG{l+s}{\PYGZsq{}}\PYG{l+s}{textbf}\PYG{l+s}{\PYGZsq{}}\PYG{p}{)}         \PYG{c}{\PYGZsh{}formatting the two columns with boldface}

\PYG{n}{rows}\PYG{p}{[}\PYG{l+m+mi}{1}\PYG{p}{]}\PYG{o}{.}\PYG{n}{mergeEmpty}\PYG{p}{(}\PYG{p}{)}
\PYG{n}{rows}\PYG{p}{[}\PYG{l+m+mi}{1}\PYG{p}{]}\PYG{o}{.}\PYG{n}{addhlines}\PYG{p}{(}\PYG{n}{title} \PYG{o}{=} \PYG{n+nb+bp}{True}\PYG{p}{)}
\PYG{n}{rows}\PYG{p}{[}\PYG{l+m+mi}{1}\PYG{p}{]}\PYG{o}{.}\PYG{n}{align}\PYG{p}{(}\PYG{p}{[}\PYG{l+s}{\PYGZsq{}}\PYG{l+s}{c}\PYG{l+s}{\PYGZsq{}}\PYG{p}{]}\PYG{o}{*}\PYG{l+m+mi}{4}\PYG{p}{)}
\PYG{n}{rows}\PYG{p}{[}\PYG{l+m+mi}{1}\PYG{p}{]}\PYG{o}{.}\PYG{n}{setFormat}\PYG{p}{(}\PYG{l+s}{\PYGZsq{}}\PYG{l+s}{emph}\PYG{l+s}{\PYGZsq{}}\PYG{p}{)}           \PYG{c}{\PYGZsh{}formatting the two columns with italics}

\PYG{n}{rows}\PYG{p}{[}\PYG{l+m+mi}{2}\PYG{p}{]}\PYG{o}{.}\PYG{n}{align}\PYG{p}{(}\PYG{p}{[}\PYG{l+s}{\PYGZsq{}}\PYG{l+s}{c}\PYG{l+s}{\PYGZsq{}}\PYG{p}{]}\PYG{o}{*}\PYG{l+m+mi}{13}\PYG{p}{)}
\PYG{n}{rows}\PYG{p}{[}\PYG{l+m+mi}{2}\PYG{p}{]}\PYG{o}{.}\PYG{n}{addhlines}\PYG{p}{(}\PYG{p}{)}                 \PYG{c}{\PYGZsh{}adding full horizontal (default title = False)}

\PYG{n}{rows}\PYG{p}{[}\PYG{l+m+mi}{3}\PYG{p}{]}\PYG{o}{.}\PYG{n}{setTopSpacing}\PYG{p}{(}\PYG{l+s}{\PYGZsq{}}\PYG{l+s}{0.25cm}\PYG{l+s}{\PYGZsq{}}\PYG{p}{)}
\PYG{n}{rows}\PYG{p}{[}\PYG{l+m+mi}{3}\PYG{p}{]}\PYG{o}{.}\PYG{n}{setBottomSpacing}\PYG{p}{(}\PYG{l+s}{\PYGZsq{}}\PYG{l+s}{0.25cm}\PYG{l+s}{\PYGZsq{}}\PYG{p}{)}

\PYG{n}{rows}\PYG{p}{[}\PYG{l+m+mi}{9}\PYG{p}{]}\PYG{o}{.}\PYG{n}{setTopSpacing}\PYG{p}{(}\PYG{l+s}{\PYGZsq{}}\PYG{l+s}{2.5mm}\PYG{l+s}{\PYGZsq{}}\PYG{p}{)}
\PYG{n}{rows}\PYG{p}{[}\PYG{l+m+mi}{9}\PYG{p}{]}\PYG{o}{.}\PYG{n}{setBottomSpacing}\PYG{p}{(}\PYG{l+s}{\PYGZsq{}}\PYG{l+s}{2.5mm}\PYG{l+s}{\PYGZsq{}}\PYG{p}{)}
\end{Verbatim}

Now let's see what we have.

\begin{Verbatim}[commandchars=\\\{\}]
\PYG{n}{texfile} \PYG{o}{=} \PYG{n+nb}{open}\PYG{p}{(}\PYG{l+s}{\PYGZsq{}}\PYG{l+s}{test3.tex}\PYG{l+s}{\PYGZsq{}}\PYG{p}{,} \PYG{l+s}{\PYGZsq{}}\PYG{l+s}{w}\PYG{l+s}{\PYGZsq{}}\PYG{p}{)}
\PYG{n}{texfile}\PYG{o}{.}\PYG{n}{write}\PYG{p}{(}\PYG{n}{table}\PYG{o}{.}\PYG{n}{makeTex}\PYG{p}{(}\PYG{p}{)}\PYG{p}{)}
\PYG{n}{texfile}\PYG{o}{.}\PYG{n}{close}\PYG{p}{(}\PYG{p}{)}
\end{Verbatim}
%Created Using: /home/jake/Documents/repos/erc/ercTools/tablemaker.py
\begin{table}[h]
\begin{center} \small
\begin{tabular}{ccccccccccccc}
\hline 

\mc{1}{c}{}&\mc{6}{c}{\textbf{Head Start Group}} & \mc{6}{c}{\textbf{Control Group}} \\
 \cmidrule(lr){2-7} \cmidrule(lr){8-13}
\mc{1}{c}{}&\mc{3}{c}{\emph{Treated}} & \mc{3}{c}{\emph{No shows}} & \mc{3}{c}{\emph{Control}} & \mc{3}{c}{\emph{Crossover}} \\
 \cmidrule(lr){2-4} \cmidrule(lr){5-7} \cmidrule(lr){8-10} \cmidrule(lr){11-13}
\mc{1}{c}{Measure} & \mc{1}{c}{obs} & \mc{1}{c}{mean} & \mc{1}{c}{std} & \mc{1}{c}{obs} & \mc{1}{c}{mean} & \mc{1}{c}{std} & \mc{1}{c}{obs} & \mc{1}{c}{mean} & \mc{1}{c}{std} & \mc{1}{c}{obs} & \mc{1}{c}{mean} & \mc{1}{c}{std} \\
\hline
\\ [0.25cm]
Parent Measures &  &  &  &  &  &  &  &  &  &  &  &  \\[0.25cm] 

Mother's education & 99 & 10.60 & 4.71 & 34 & 6.27 & 4.42 & 33 & 6.20 & 4.41 & 34 & 6.27 & 4.42 \\

Father's education & 100 & 10.67 & 4.71 & 35 & 6.33 & 4.42 & 34 & 6.27 & 4.42 & 35 & 6.33 & 4.42 \\

Combined education & 101 & 10.73 & 4.72 & 36 & 6.40 & 4.43 & 35 & 6.33 & 4.42 & 36 & 6.40 & 4.43 \\

Mother's occupation level & 102 & 10.80 & 4.72 & 37 & 6.47 & 4.43 & 36 & 6.40 & 4.43 & 37 & 6.47 & 4.43 \\

Father's occupation level & 103 & 10.87 & 4.72 & 38 & 6.53 & 4.44 & 37 & 6.47 & 4.43 & 38 & 6.53 & 4.44 \\

\\ [2.5mm]
Child Measures &  &  &  &  &  &  &  &  &  &  &  &  \\[2.5mm] 

IQ Score & 99 & 10.60 & 4.71 & 34 & 6.27 & 4.42 & 56 & 7.73 & 4.52 & 56 & 7.73 & 4.52 \\

Behavior Index & 100 & 10.67 & 4.71 & 35 & 6.33 & 4.42 & 57 & 7.80 & 4.52 & 57 & 7.80 & 4.52 \\

PPVT Score & 101 & 10.73 & 4.72 & 36 & 6.40 & 4.43 & 58 & 7.87 & 4.52 & 58 & 7.87 & 4.52 \\

Timeouts & 102 & 10.80 & 4.72 & 37 & 6.47 & 4.43 & 59 & 7.93 & 4.53 & 59 & 7.93 & 4.53 \\

\hline 
\end{tabular}
\end{center}
\end{table}
And now what about adding footnotes, labels, and a title? We can also provide different
default parameters for LateX's tabular environment.

\begin{Verbatim}[commandchars=\\\{\}]
\PYG{n}{table}\PYG{o}{.}\PYG{n}{setTitle}\PYG{p}{(}\PYG{l+s}{\PYGZsq{}}\PYG{l+s}{Parent and Child Measures of Head Start Families, by Group}\PYG{l+s}{\PYGZsq{}}\PYG{p}{)}
\PYG{n}{table}\PYG{o}{.}\PYG{n}{setLabel}\PYG{p}{(}\PYG{l+s}{\PYGZsq{}}\PYG{l+s}{hs\PYGZus{}label}\PYG{l+s}{\PYGZsq{}}\PYG{p}{)}
\PYG{n}{table}\PYG{o}{.}\PYG{n}{setNotes}\PYG{p}{(}\PYG{l+s}{\PYGZsq{}}\PYG{l+s}{These statistics were not weighted using HSIS longitudinal weights.}\PYG{l+s}{\PYGZsq{}}\PYG{p}{)}
\PYG{n}{table}\PYG{o}{.}\PYG{n}{setTableAlignment}\PYG{p}{(}\PYG{l+s}{\PYGZsq{}}\PYG{l+s}{lllllllllllll}\PYG{l+s}{\PYGZsq{}}\PYG{p}{)}                        \PYG{c}{\PYGZsh{}left justifying table by default}

\PYG{n}{texfile} \PYG{o}{=} \PYG{n+nb}{open}\PYG{p}{(}\PYG{l+s}{\PYGZsq{}}\PYG{l+s}{test4.tex}\PYG{l+s}{\PYGZsq{}}\PYG{p}{,} \PYG{l+s}{\PYGZsq{}}\PYG{l+s}{w}\PYG{l+s}{\PYGZsq{}}\PYG{p}{)}
\PYG{n}{texfile}\PYG{o}{.}\PYG{n}{write}\PYG{p}{(}\PYG{n}{table}\PYG{o}{.}\PYG{n}{makeTex}\PYG{p}{(}\PYG{p}{)}\PYG{p}{)}
\PYG{n}{texfile}\PYG{o}{.}\PYG{n}{close}\PYG{p}{(}\PYG{p}{)}
\end{Verbatim}
%Created Using: /home/jake/Documents/repos/erc/ercTools/tablemaker.py
\begin{table}[h]
\caption{Parent and Child Measures of Head Start Families, by Group}
\label{hs_label}
\begin{center} \small
\begin{tabular}{lllllllllllll}
\hline 

\mc{1}{c}{}&\mc{6}{c}{\textbf{Head Start Group}} & \mc{6}{c}{\textbf{Control Group}} \\
 \cmidrule(lr){2-7} \cmidrule(lr){8-13}
\mc{1}{c}{}&\mc{3}{c}{\emph{Treated}} & \mc{3}{c}{\emph{No shows}} & \mc{3}{c}{\emph{Control}} & \mc{3}{c}{\emph{Crossover}} \\
 \cmidrule(lr){2-4} \cmidrule(lr){5-7} \cmidrule(lr){8-10} \cmidrule(lr){11-13}
\mc{1}{c}{Measure} & \mc{1}{c}{obs} & \mc{1}{c}{mean} & \mc{1}{c}{std} & \mc{1}{c}{obs} & \mc{1}{c}{mean} & \mc{1}{c}{std} & \mc{1}{c}{obs} & \mc{1}{c}{mean} & \mc{1}{c}{std} & \mc{1}{c}{obs} & \mc{1}{c}{mean} & \mc{1}{c}{std} \\
\hline
\\ [0.25cm]
Parent Measures &  &  &  &  &  &  &  &  &  &  &  &  \\[0.25cm] 

Mother's education & 99 & 10.60 & 4.71 & 34 & 6.27 & 4.42 & 33 & 6.20 & 4.41 & 34 & 6.27 & 4.42 \\

Father's education & 100 & 10.67 & 4.71 & 35 & 6.33 & 4.42 & 34 & 6.27 & 4.42 & 35 & 6.33 & 4.42 \\

Combined education & 101 & 10.73 & 4.72 & 36 & 6.40 & 4.43 & 35 & 6.33 & 4.42 & 36 & 6.40 & 4.43 \\

Mother's occupation level & 102 & 10.80 & 4.72 & 37 & 6.47 & 4.43 & 36 & 6.40 & 4.43 & 37 & 6.47 & 4.43 \\

Father's occupation level & 103 & 10.87 & 4.72 & 38 & 6.53 & 4.44 & 37 & 6.47 & 4.43 & 38 & 6.53 & 4.44 \\

\\ [2.5mm]
Child Measures &  &  &  &  &  &  &  &  &  &  &  &  \\[2.5mm] 

IQ Score & 99 & 10.60 & 4.71 & 34 & 6.27 & 4.42 & 56 & 7.73 & 4.52 & 56 & 7.73 & 4.52 \\

Behavior Index & 100 & 10.67 & 4.71 & 35 & 6.33 & 4.42 & 57 & 7.80 & 4.52 & 57 & 7.80 & 4.52 \\

PPVT Score & 101 & 10.73 & 4.72 & 36 & 6.40 & 4.43 & 58 & 7.87 & 4.52 & 58 & 7.87 & 4.52 \\

Timeouts & 102 & 10.80 & 4.72 & 37 & 6.47 & 4.43 & 59 & 7.93 & 4.53 & 59 & 7.93 & 4.53 \\

\hline 
\end{tabular}
\end{center}
\scriptsize{	extbf{Notes:}
These statistics were not weighted using HSIS longitudinal weights.}
\end{table}
Now what if we have a longer table?
Sheet2 is a long table that doesn't fit on one page

\begin{Verbatim}[commandchars=\\\{\}]
\PYG{n}{table} \PYG{o}{=} \PYG{n}{tm}\PYG{o}{.}\PYG{n}{Table}\PYG{p}{(}\PYG{n}{sheetDict}\PYG{p}{[}\PYG{l+s}{\PYGZsq{}}\PYG{l+s}{Sheet2}\PYG{l+s}{\PYGZsq{}}\PYG{p}{]}\PYG{p}{)}

\PYG{n}{texfile} \PYG{o}{=} \PYG{n+nb}{open}\PYG{p}{(}\PYG{l+s}{\PYGZsq{}}\PYG{l+s}{test5.tex}\PYG{l+s}{\PYGZsq{}}\PYG{p}{,} \PYG{l+s}{\PYGZsq{}}\PYG{l+s}{w}\PYG{l+s}{\PYGZsq{}}\PYG{p}{)}
\PYG{n}{texfile}\PYG{o}{.}\PYG{n}{write}\PYG{p}{(}\PYG{n}{table}\PYG{o}{.}\PYG{n}{makeTex}\PYG{p}{(}\PYG{p}{)}\PYG{p}{)}
\PYG{n}{texfile}\PYG{o}{.}\PYG{n}{close}\PYG{p}{(}\PYG{p}{)}
\end{Verbatim}
%Created Using: /home/jake/Documents/repos/erc/ercTools/tablemaker.py
\begin{table}[h]
\begin{center} \small
\begin{tabular}{ccc}
\hline 

variable & mean & std \\

var & 34 & 1 \\

var & 36.50 & 1.07 \\

var & 39 & 1.15 \\

var & 41.50 & 1.22 \\

var & 44 & 1.29 \\

var & 46.50 & 1.37 \\

var & 49 & 1.44 \\

var & 51.50 & 1.51 \\

var & 54 & 1.59 \\

var & 56.50 & 1.66 \\

var & 59 & 1.74 \\

var & 61.50 & 1.81 \\

var & 64 & 1.88 \\

var & 66.50 & 1.96 \\

var & 69 & 2.03 \\

var & 71.50 & 2.10 \\

var & 74 & 2.18 \\

var & 76.50 & 2.25 \\

var & 79 & 2.32 \\

var & 81.50 & 2.40 \\

var & 84 & 2.47 \\

var & 86.50 & 2.54 \\

var & 89 & 2.62 \\

var & 91.50 & 2.69 \\

var & 94 & 2.76 \\

var & 96.50 & 2.84 \\

var & 99 & 2.91 \\

var & 101.50 & 2.99 \\

var & 104 & 3.06 \\

var & 106.50 & 3.13 \\

var & 109 & 3.21 \\

var & 111.50 & 3.28 \\

var & 114 & 3.35 \\

var & 116.50 & 3.43 \\

var & 119 & 3.50 \\

var & 121.50 & 3.57 \\

var & 124 & 3.65 \\

var & 126.50 & 3.72 \\

var & 129 & 3.79 \\

var & 131.50 & 3.87 \\

var & 134 & 3.94 \\

var & 136.50 & 4.01 \\

var & 139 & 4.09 \\

var & 141.50 & 4.16 \\

var & 144 & 4.24 \\

var & 146.50 & 4.31 \\

var & 149 & 4.38 \\

var & 151.50 & 4.46 \\

var & 154 & 4.53 \\

var & 156.50 & 4.60 \\

var & 159 & 4.68 \\

var & 161.50 & 4.75 \\

var & 164 & 4.82 \\

var & 166.50 & 4.90 \\

var & 169 & 4.97 \\

var & 171.50 & 5.04 \\

var & 174 & 5.12 \\

var & 176.50 & 5.19 \\

var & 179 & 5.26 \\

var & 181.50 & 5.34 \\

var & 184 & 5.41 \\

var & 186.50 & 5.49 \\

var & 189 & 5.56 \\

var & 191.50 & 5.63 \\

var & 194 & 5.71 \\

var & 196.50 & 5.78 \\

var & 199 & 5.85 \\

var & 201.50 & 5.93 \\

var & 204 & 6 \\

var & 206.50 & 6.07 \\

var & 209 & 6.15 \\

var & 211.50 & 6.22 \\

var & 214 & 6.29 \\

var & 216.50 & 6.37 \\

var & 219 & 6.44 \\

var & 221.50 & 6.51 \\

var & 224 & 6.59 \\

var & 226.50 & 6.66 \\

var & 229 & 6.74 \\

var & 231.50 & 6.81 \\

var & 234 & 6.88 \\

var & 236.50 & 6.96 \\

var & 239 & 7.03 \\

var & 241.50 & 7.10 \\

var & 244 & 7.18 \\

var & 246.50 & 7.25 \\

var & 249 & 7.32 \\

\hline 
\end{tabular}
\end{center}
\end{table}
\clearpage
If we specify the type as longtable, the table will flow onto
multiple pages.

\begin{Verbatim}[commandchars=\\\{\}]
\PYG{n}{table}\PYG{o}{.}\PYG{n}{setType}\PYG{p}{(}\PYG{l+s}{\PYGZsq{}}\PYG{l+s}{longtable}\PYG{l+s}{\PYGZsq{}}\PYG{p}{)}
\PYG{n}{table}\PYG{o}{.}\PYG{n}{setTitle}\PYG{p}{(}\PYG{l+s}{\PYGZsq{}}\PYG{l+s}{This is a Long Table}\PYG{l+s}{\PYGZsq{}}\PYG{p}{)}
\PYG{n}{table}\PYG{o}{.}\PYG{n}{setRepeat}\PYG{p}{(}\PYG{l+m+mi}{0}\PYG{p}{)}      \PYG{c}{\PYGZsh{}row number(s) to repeat on every page}
\PYG{n}{table}\PYG{o}{.}\PYG{n}{setHlines}\PYG{p}{(}\PYG{p}{(}\PYG{l+m+mi}{0}\PYG{p}{,}\PYG{l+m+mi}{0}\PYG{p}{)}\PYG{p}{)}  \PYG{c}{\PYGZsh{}eliminating default top and bottom lines}

\PYG{n}{rows} \PYG{o}{=} \PYG{n}{table}\PYG{o}{.}\PYG{n}{rows}

\PYG{n}{rows}\PYG{p}{[}\PYG{l+m+mi}{0}\PYG{p}{]}\PYG{o}{.}\PYG{n}{addhlines}\PYG{p}{(}\PYG{p}{)}

\PYG{n}{texfile} \PYG{o}{=} \PYG{n+nb}{open}\PYG{p}{(}\PYG{l+s}{\PYGZsq{}}\PYG{l+s}{test6.tex}\PYG{l+s}{\PYGZsq{}}\PYG{p}{,} \PYG{l+s}{\PYGZsq{}}\PYG{l+s}{w}\PYG{l+s}{\PYGZsq{}}\PYG{p}{)}
\PYG{n}{texfile}\PYG{o}{.}\PYG{n}{write}\PYG{p}{(}\PYG{n}{table}\PYG{o}{.}\PYG{n}{makeTex}\PYG{p}{(}\PYG{p}{)}\PYG{p}{)}
\PYG{n}{texfile}\PYG{o}{.}\PYG{n}{close}\PYG{p}{(}\PYG{p}{)}
\end{Verbatim}
\clearpage
%Created Using: /home/jake/Documents/repos/erc/ercTools/tablemaker.py

\begin{center}
\small
\begin{longtable}[h]{ccc}

\caption{This is a Long Table}\\
variable & mean & std \\
\hline
\endfirsthead
\mc{3}{c}%
{\tablename\ \thetable\ -- \emph{Continued from previous page}} \\
variable & mean & std \\
\hline

\endhead
\mc{3}{r}{\emph{Continued on next page}} \\
\endfoot
\endlastfoot
var & 34 & 1 \\

var & 36.50 & 1.07 \\

var & 39 & 1.15 \\

var & 41.50 & 1.22 \\

var & 44 & 1.29 \\

var & 46.50 & 1.37 \\

var & 49 & 1.44 \\

var & 51.50 & 1.51 \\

var & 54 & 1.59 \\

var & 56.50 & 1.66 \\

var & 59 & 1.74 \\

var & 61.50 & 1.81 \\

var & 64 & 1.88 \\

var & 66.50 & 1.96 \\

var & 69 & 2.03 \\

var & 71.50 & 2.10 \\

var & 74 & 2.18 \\

var & 76.50 & 2.25 \\

var & 79 & 2.32 \\

var & 81.50 & 2.40 \\

var & 84 & 2.47 \\

var & 86.50 & 2.54 \\

var & 89 & 2.62 \\

var & 91.50 & 2.69 \\

var & 94 & 2.76 \\

var & 96.50 & 2.84 \\

var & 99 & 2.91 \\

var & 101.50 & 2.99 \\

var & 104 & 3.06 \\

var & 106.50 & 3.13 \\

var & 109 & 3.21 \\

var & 111.50 & 3.28 \\

var & 114 & 3.35 \\

var & 116.50 & 3.43 \\

var & 119 & 3.50 \\

var & 121.50 & 3.57 \\

var & 124 & 3.65 \\

var & 126.50 & 3.72 \\

var & 129 & 3.79 \\

var & 131.50 & 3.87 \\

var & 134 & 3.94 \\

var & 136.50 & 4.01 \\

var & 139 & 4.09 \\

var & 141.50 & 4.16 \\

var & 144 & 4.24 \\

var & 146.50 & 4.31 \\

var & 149 & 4.38 \\

var & 151.50 & 4.46 \\

var & 154 & 4.53 \\

var & 156.50 & 4.60 \\

var & 159 & 4.68 \\

var & 161.50 & 4.75 \\

var & 164 & 4.82 \\

var & 166.50 & 4.90 \\

var & 169 & 4.97 \\

var & 171.50 & 5.04 \\

var & 174 & 5.12 \\

var & 176.50 & 5.19 \\

var & 179 & 5.26 \\

var & 181.50 & 5.34 \\

var & 184 & 5.41 \\

var & 186.50 & 5.49 \\

var & 189 & 5.56 \\

var & 191.50 & 5.63 \\

var & 194 & 5.71 \\

var & 196.50 & 5.78 \\

var & 199 & 5.85 \\

var & 201.50 & 5.93 \\

var & 204 & 6 \\

var & 206.50 & 6.07 \\

var & 209 & 6.15 \\

var & 211.50 & 6.22 \\

var & 214 & 6.29 \\

var & 216.50 & 6.37 \\

var & 219 & 6.44 \\

var & 221.50 & 6.51 \\

var & 224 & 6.59 \\

var & 226.50 & 6.66 \\

var & 229 & 6.74 \\

var & 231.50 & 6.81 \\

var & 234 & 6.88 \\

var & 236.50 & 6.96 \\

var & 239 & 7.03 \\

var & 241.50 & 7.10 \\

var & 244 & 7.18 \\

var & 246.50 & 7.25 \\

var & 249 & 7.32 \\


\end{longtable}
\end{center}

\clearpage
We can also construct a table without an .xls file.
Let's assume we have some data.

\begin{Verbatim}[commandchars=\\\{\}]
\PYG{k+kn}{import} \PYG{n+nn}{pandas} \PYG{k+kn}{as} \PYG{n+nn}{pd}
\PYG{k+kn}{import} \PYG{n+nn}{numpy} \PYG{k+kn}{as} \PYG{n+nn}{np}

\PYG{c}{\PYGZsh{}generate some random data}
\PYG{n}{data} \PYG{o}{=} \PYG{n}{np}\PYG{o}{.}\PYG{n}{random}\PYG{o}{.}\PYG{n}{randn}\PYG{p}{(}\PYG{l+m+mi}{1000}\PYG{p}{)}
\PYG{n}{data} \PYG{o}{=} \PYG{n}{data}\PYG{o}{.}\PYG{n}{reshape}\PYG{p}{(}\PYG{p}{(}\PYG{l+m+mi}{100}\PYG{p}{,}\PYG{l+m+mi}{10}\PYG{p}{)}\PYG{p}{)}   \PYG{c}{\PYGZsh{}100 rows and 10 columns}
\PYG{n}{dataframe} \PYG{o}{=} \PYG{n}{pd}\PYG{o}{.}\PYG{n}{DataFrame}\PYG{p}{(}\PYG{n}{data}\PYG{p}{)}

\PYG{c}{\PYGZsh{}fake names for variables}
\PYG{n}{dataframe}\PYG{o}{.}\PYG{n}{columns} \PYG{o}{=} \PYG{p}{[}\PYG{l+s}{\PYGZsq{}}\PYG{l+s}{var}\PYG{l+s}{\PYGZsq{}} \PYG{o}{+} \PYG{n+nb}{str}\PYG{p}{(}\PYG{n}{i}\PYG{p}{)} \PYG{k}{for} \PYG{n}{i} \PYG{o+ow}{in} \PYG{n+nb}{range}\PYG{p}{(}\PYG{l+m+mi}{1}\PYG{p}{,}\PYG{l+m+mi}{11}\PYG{p}{)}\PYG{p}{]}

\PYG{k}{print} \PYG{n}{dataframe}\PYG{p}{[}\PYG{p}{[}\PYG{l+s}{\PYGZsq{}}\PYG{l+s}{var1}\PYG{l+s}{\PYGZsq{}}\PYG{p}{,} \PYG{l+s}{\PYGZsq{}}\PYG{l+s}{var2}\PYG{l+s}{\PYGZsq{}}\PYG{p}{,} \PYG{l+s}{\PYGZsq{}}\PYG{l+s}{var3}\PYG{l+s}{\PYGZsq{}}\PYG{p}{]}\PYG{p}{]}\PYG{o}{.}\PYG{n}{head}\PYG{p}{(}\PYG{l+m+mi}{10}\PYG{p}{)}  \PYG{c}{\PYGZsh{}take a look at data}
\end{Verbatim}

Let's create a table of means and standard deviations of these variables.

\begin{Verbatim}[commandchars=\\\{\}]
\PYG{n}{rowdict} \PYG{o}{=} \PYG{p}{\PYGZob{}}\PYG{p}{\PYGZcb{}}
\PYG{n}{rowdict}\PYG{p}{[}\PYG{l+m+mi}{0}\PYG{p}{]} \PYG{o}{=} \PYG{p}{[}\PYG{l+s}{\PYGZsq{}}\PYG{l+s}{Measure}\PYG{l+s}{\PYGZsq{}}\PYG{p}{,} \PYG{l+s}{\PYGZsq{}}\PYG{l+s}{Mean}\PYG{l+s}{\PYGZsq{}}\PYG{p}{,} \PYG{l+s}{\PYGZsq{}}\PYG{l+s}{Std}\PYG{l+s}{\PYGZsq{}}\PYG{p}{]}
\PYG{n}{row} \PYG{o}{=} \PYG{l+m+mi}{1}

\PYG{k}{for} \PYG{n}{i} \PYG{o+ow}{in} \PYG{n+nb}{range}\PYG{p}{(}\PYG{l+m+mi}{1}\PYG{p}{,}\PYG{l+m+mi}{11}\PYG{p}{)}\PYG{p}{:}
    \PYG{n}{varname} \PYG{o}{=} \PYG{l+s}{\PYGZsq{}}\PYG{l+s}{var}\PYG{l+s}{\PYGZsq{}} \PYG{o}{+} \PYG{n+nb}{str}\PYG{p}{(}\PYG{n}{i}\PYG{p}{)}
    \PYG{n}{series} \PYG{o}{=} \PYG{n}{dataframe}\PYG{p}{[}\PYG{n}{varname}\PYG{p}{]}
    \PYG{n}{mean} \PYG{o}{=} \PYG{n}{series}\PYG{o}{.}\PYG{n}{mean}\PYG{p}{(}\PYG{p}{)}
    \PYG{n}{std} \PYG{o}{=} \PYG{n}{series}\PYG{o}{.}\PYG{n}{std}\PYG{p}{(}\PYG{p}{)}
    \PYG{n}{rowdict}\PYG{p}{[}\PYG{n}{row}\PYG{p}{]} \PYG{o}{=} \PYG{p}{[}\PYG{n}{varname}\PYG{p}{,} \PYG{n}{mean}\PYG{p}{,} \PYG{n}{std}\PYG{p}{]}

    \PYG{n}{row} \PYG{o}{+}\PYG{o}{=} \PYG{l+m+mi}{1}
\end{Verbatim}

Now the data for the table is stored in a dictionary similar to the one
produced from the tm.read() method. So we can instantiate a table object.

\begin{Verbatim}[commandchars=\\\{\}]
\PYG{n}{table} \PYG{o}{=} \PYG{n}{tm}\PYG{o}{.}\PYG{n}{Table}\PYG{p}{(}\PYG{n}{rowdict}\PYG{p}{)}

\PYG{n}{table}\PYG{o}{.}\PYG{n}{setTitle}\PYG{p}{(}\PYG{l+s}{\PYGZsq{}}\PYG{l+s}{This table was created without a .xls file}\PYG{l+s}{\PYGZsq{}}\PYG{p}{)}

\PYG{n}{texfile} \PYG{o}{=} \PYG{n+nb}{open}\PYG{p}{(}\PYG{l+s}{\PYGZsq{}}\PYG{l+s}{test7.tex}\PYG{l+s}{\PYGZsq{}}\PYG{p}{,} \PYG{l+s}{\PYGZsq{}}\PYG{l+s}{w}\PYG{l+s}{\PYGZsq{}}\PYG{p}{)}
\PYG{n}{texfile}\PYG{o}{.}\PYG{n}{write}\PYG{p}{(}\PYG{n}{table}\PYG{o}{.}\PYG{n}{makeTex}\PYG{p}{(}\PYG{p}{)}\PYG{p}{)}
\PYG{n}{texfile}\PYG{o}{.}\PYG{n}{close}\PYG{p}{(}\PYG{p}{)}
\end{Verbatim}

You can use the methods we have discussed previously to modify the
table to your liking, but present here the basic table to prove
our test worked.
%Created Using: /home/jake/Documents/repos/erc/ercTools/tablemaker.py
\begin{table}[h]
\caption{This table was created without a .xls file}
\begin{center} \small
\begin{tabular}{ccc}
\hline 

Measure & Mean & Std \\

var1 & -0.10 & 1.04 \\

var2 & 0.10 & 1 \\

var3 & -0.11 & 1.11 \\

var4 & 0.04 & 0.89 \\

var5 & -0.07 & 1.02 \\

var6 & -0.01 & 1.03 \\

var7 & -0.21 & 1.09 \\

var8 & -0.11 & 0.97 \\

var9 & -0.15 & 0.98 \\

var10 & -0.04 & 0.90 \\

\hline 
\end{tabular}
\end{center}
\end{table}
\clearpage
Does tablemaker support multirow functionality? Yes! Let's try it out!

\begin{Verbatim}[commandchars=\\\{\}]
\PYG{n}{rowdict} \PYG{o}{=}       \PYG{p}{\PYGZob{}}\PYG{l+m+mi}{0}\PYG{p}{:}\PYG{p}{[}\PYG{l+s}{\PYGZsq{}}\PYG{l+s}{A multiple row column}\PYG{l+s}{\PYGZsq{}}\PYG{p}{,} \PYG{l+s}{\PYGZsq{}}\PYG{l+s}{B1}\PYG{l+s}{\PYGZsq{}}\PYG{p}{,} \PYG{l+s}{\PYGZsq{}}\PYG{l+s}{C1}\PYG{l+s}{\PYGZsq{}}\PYG{p}{]}\PYG{p}{,}
                         \PYG{l+m+mi}{1}\PYG{p}{:}\PYG{p}{[}\PYG{l+s}{\PYGZsq{}}\PYG{l+s}{\PYGZsq{}}\PYG{p}{,} \PYG{l+s}{\PYGZsq{}}\PYG{l+s}{B2}\PYG{l+s}{\PYGZsq{}}\PYG{p}{,} \PYG{l+s}{\PYGZsq{}}\PYG{l+s}{C2}\PYG{l+s}{\PYGZsq{}}\PYG{p}{]}\PYG{p}{,}
                         \PYG{l+m+mi}{2}\PYG{p}{:}\PYG{p}{[}\PYG{l+s}{\PYGZsq{}}\PYG{l+s}{\PYGZsq{}}\PYG{p}{,} \PYG{l+s}{\PYGZsq{}}\PYG{l+s}{B3}\PYG{l+s}{\PYGZsq{}}\PYG{p}{,} \PYG{l+s}{\PYGZsq{}}\PYG{l+s}{C3}\PYG{l+s}{\PYGZsq{}}\PYG{p}{]}\PYG{p}{\PYGZcb{}}
\PYG{n}{table} \PYG{o}{=} \PYG{n}{tm}\PYG{o}{.}\PYG{n}{Table}\PYG{p}{(}\PYG{n}{rowdict}\PYG{p}{)}
\PYG{n}{table}\PYG{o}{.}\PYG{n}{setTitle}\PYG{p}{(}\PYG{l+s}{\PYGZsq{}}\PYG{l+s}{A table with multirow functionality}\PYG{l+s}{\PYGZsq{}}\PYG{p}{)}

\PYG{n}{rows} \PYG{o}{=} \PYG{n}{table}\PYG{o}{.}\PYG{n}{rows}
\PYG{n}{rows}\PYG{p}{[}\PYG{l+m+mi}{0}\PYG{p}{]}\PYG{o}{.}\PYG{n}{setFormat}\PYG{p}{(}\PYG{p}{[}\PYG{l+s}{\PYGZsq{}}\PYG{l+s}{multirow\PYGZob{}3\PYGZcb{}\PYGZob{}*\PYGZcb{}}\PYG{l+s}{\PYGZsq{}}\PYG{p}{,} \PYG{l+s}{\PYGZsq{}}\PYG{l+s}{\PYGZsq{}}\PYG{p}{,} \PYG{l+s}{\PYGZsq{}}\PYG{l+s}{\PYGZsq{}}\PYG{p}{]}\PYG{p}{)}

\PYG{n}{texfile} \PYG{o}{=} \PYG{n+nb}{open}\PYG{p}{(}\PYG{l+s}{\PYGZsq{}}\PYG{l+s}{test8.tex}\PYG{l+s}{\PYGZsq{}}\PYG{p}{,} \PYG{l+s}{\PYGZsq{}}\PYG{l+s}{w}\PYG{l+s}{\PYGZsq{}}\PYG{p}{)}
\PYG{n}{texfile}\PYG{o}{.}\PYG{n}{write}\PYG{p}{(}\PYG{n}{table}\PYG{o}{.}\PYG{n}{makeTex}\PYG{p}{(}\PYG{p}{)}\PYG{p}{)}
\PYG{n}{texfile}\PYG{o}{.}\PYG{n}{close}\PYG{p}{(}\PYG{p}{)}
\end{Verbatim}
%Created Using: /home/jake/Documents/repos/erc/ercTools/tablemaker.py
\begin{table}[h]
\caption{A table with multirow functionality}
\begin{center} \small
\begin{tabular}{ccc}
\hline 

\multirow{3}{*}{A multiple row column} & B1 & C1 \\

 & B2 & C2 \\

 & B3 & C3 \\

\hline 
\end{tabular}
\end{center}
\end{table}
\clearpage

\renewcommand{\indexname}{Python Module Index}
\begin{theindex}
\def\bigletter#1{{\Large\sffamily#1}\nopagebreak\vspace{1mm}}
\bigletter{t}
\item {\texttt{tablemaker}}, \pageref{summary:module-tablemaker}
\end{theindex}

\renewcommand{\indexname}{Index}
\printindex
\end{document}
