\documentclass{article}
\usepackage{geometry}
\usepackage{hyperref}

\newgeometry{left=2cm, right=2cm}

\begin{document}
Here I'll provide some steps for installing ubuntu 13.04 and configuring it for
use at the Economics Research Center. Keep in mind through not to copy and paste commands, as copying from the pdf might not copy the correct characters. Also, whenever a double-dash appears like ``- -'', you should
type two consecutive dashes. 

\begin{enumerate}
\item First create a bootable usb drive containing from the appropriate .iso install file
and a usb/startup disk creator. Instructions on how to create this drive are provided on
the \href{http://www.ubuntu.com/download/desktop}{ubuntu website}. 	

\item Then reboot your computer with the usb stick attached, open the boot options menu
during startup and select the usb drive. 

\item You are now booting from the usb. Follow the instructions for installation. Make
sure, if you plan on duel booting, to ``install ubuntu alongside Windows''. Also make
sure to install updates and 3rd party software while installing ubuntu. 

\item Now that you've installed Ubuntu it's time to configure it so you can get to work. First
go into terminal (Ctrl + Alt + T) and type ``sudo apt-get update'' to update your machine once again. The ``sudo'' command is used to give permissions to the current user because in some cases a command requires the permissions of another user (like root or admin). If you ever see an error when you use a command and it has to do with permissions, try using ``sudo'' to run the command as root. First, let's 
install R onto your computer. Enter the following commands into terminal:
	
	\begin{itemize}
		 \item gpg - -keyserver keyserver.ubuntu.com - -recv-key E084DAB9
		 \item gpg -a - -export E084DAB9 $\mid$ sudo apt-key add -
		 \item sudo gedit /etc/apt/sources.list

		 and append the following to the end of this file:

		 deb http://cran.cnr.berkeley.edu/bin/linux/ubuntu precise/

		 Essentially, you have just told the ``apt-get'' package to look for 
		 package updates for R at this location. I chose Berkeley because
		 they should reguarly update the scientific packages we use.

		 \item sudo apt-get update
		 \item sudo apt-get install r-base
		 \item R should now work, type ``R'' in the terminal to see if the R console comes up. 

	\end{itemize}
\item Now install the rpy2 package (a package for creating R objects in Python), type the following command:

sudo apt-get install python-rpy2

\item Now let's map your computer's drive system to athens. NOTE: This is after you gain access to the University Network, 
if you are at home you would first need to vpn into the network. I will explain how to do this in the next step.

	\begin{itemize}
		\item  sudo gedit /etc/fstab

		Modifying this file will tell your computer what drives to mount. Add the following text to the
		end of this document: ``//128.135.47.237/klmshare /home/$<$UbuntuUser$>$/Klmshare cifs username=$<$Athens ID$>$,password=$<$Athens Password$>$,uid=$<$UbuntuUser$>$,defaults 0 0 ''
		Make sure to substitute your appropriate username and passwords for $<$UbuntuUser$>$, $<$Athens ID$>$, and $<$Athens Password$>$. The $<$Athens ID$>$ and $<$Athens Password$>$ are the
		username and password you would use to access athens. $<$UbuntuUser$>$ is your ubuntu user name. It is the name before the ``@'' symbol in terminal. 
		
		\item mkdir $\sim$/Klmshare

		\item sudo apt-get install cifs-utils

		Now if you remount your drives you should mount klmshare. 

		\item sudo mount -a

	\end{itemize}

\item Let's setup vpn (for those users on a machine they use outside the University of Chicago campus)

	\begin{itemize}
		\item First, request the service if you have not already done so \href{https://cvpn.uchicago.edu/}{here}. Login to the website. 
		\item Download the cisco vpn client. It puts a shell script into your Downloads folder. To run the installation 
		script type the following command into terminal: ``sh $\sim$/Downloads/vpnsetup.sh''.
		\item Now open Cisco VPN Client and connect to ``cvpn.uchicago.edu''.
		\item After you vpn into the network you will be able to mount the klmshare folder to your computer. 
	\end{itemize}

\item Editing your environment variables. We use two environment variables at the Economics Research Center. One is called
``klmshare'' and the other is ``erc''. ``klmshare'' is the path to your mounted Klmshare drive: ``/home/$<$UbuntuUser$>$/Klmshare''. 
``erc'' is the path to your root folder for work in the erc. Within the ``erc'' folder you should put the folder ``ercprojects''. To create
directories at these locations use the ``mkdir $<$foldername$>$'' command. To create these environment variables follow these steps:

	\begin{itemize}
		\item sudo gedit $\sim$/.pam\_environment

		Then add the following text:
		\begin{tabular}{l}
			rpath DEFAULT=/usr/bin/Rscript \\
			statapath DEFAULT=/usr/local/stata12/xstata-mp \\
			klmshare DEFAULT=/home/$<$UbuntuUser$>$/Klmshare \\
			erc DEFAULT=$<$location of erc folder$>$ \\
		\end{tabular}

		\item restart your computer for these environment variables to take effect. To test
		an environment variable simply type ``echo \$$<$name of variable$>$'' and see if the terminal
		returns the correct path from the previous step. 
	\end{itemize}

\item Installing STATA. 

	\begin{itemize}
		\item Make sure you are mounted to Klmshare.
		\item sudo mkdir /usr/local/stata12
		\item cd /usr/local/stata12
		\item Next you will run the install script. In the install options
		make sure you use stata for linux when prompted. 
		\item $\sim$/Klmshare/Stata12/install
		\item Then follow the instructions for installation (you should install stata-mp)
		\item After installation is will ask you to run stinit. Run this
		using ``sudo ./stinit''
		\item then create the following desktop file, by first typing ``sudo gedit /usr/share/applications/stata12.desktop'', 
		and then adding the following to the file:
		\begin{tabular}{l}
			[Desktop Entry] \\
			Version= 12.1 \\
			Terminal=false \\
			Icon=/usr/local/stata12/stata12.png \\
			Type=Application \\
			Categories=Education;Scientific; \\
			Exec=/usr/local/stata12/xstata-mp \\
			MimeType=application/x-stata-dta;application/x-stata-do; \\
			Name=Stata/MP 12 \\
			Comment=Perform statistical analyses using Stata. \\
		\end{tabular}

		\item If the installation worked properly you should be able to open stata by typing ``\$statapath''. 
	\end{itemize}

\item Installing and configuring Eclipse (IDE)

	\begin{itemize}
		\item open up ``Ubuntu Software Center''
		\item search for ``eclipse'' and install the version that says something like ``extensible tool platform and ide''
		\item Once installed, launch eclipse and we will install the PyDev plugin (for Python). 
		\item Navigate to Help -> Install New Software. In work with, type: ``PyDev - http://pydev.org/updates'' and press ``add...''
		\item then check the first option, ``PyDev'', and click next. Accept the agreement. 
		\item When the ``Selection Needed'' dialogue box appears, select the only entry and select ``ok''. 
		\item Now we will tell eclipse which Interpreter to use for python. go to Window -> Preferences, expand PyDev and select
		Interpreter-Python. Click ``New...'' and type ``Python27''. Then for the path of the executable, type ``/usr/local/bin/python2.7''. 
		\item Then click ``ok'' until done. 
		\item To test the configuration restart eclipse and go to File->New->Project. Select PyDev->PyDev project. Create a project called ``HelloWorld'' and
		select the interpreter name you just created, ``Python27''. Tell it not to configure the PYTHONPATH. Then click ``finish''. 
		\item change the perspective to PyDev (Window->Open Perspective->Other) and select PyDev. 
		\item right-click on the HelloWorld folder on the right and create a new file called ``HelloWorld.py''. Then add ``print ``Hello World!'''' to the file. 
		\item Hit the green play button, at the bottom, ``Hello World!'' should appear. 
		\item If you want to add the C plugin do the following. 
		\item Go to help->install software. Type ``http://download.eclipse.org/tools/cdt/releases/helios'' and then click ``add...''. Follow similar steps to PyDev. 
		\item to test the installation create a c project called ``HelloWorld''. Create a file in it called ``HelloWorld.c''. Then put the following code into the file:
		\begin{tabular}{l}
			$\#$include$<$stdio.h$>$ \\
			int main() \{ \\
				printf(``Hello World!''); \\
				return 0; \\
			\} \\
		\end{tabular}
		\item Then right click on the project folder and click ``Build Project'', then go to Run As -> Local C/C++ Application. You should hopefully see ``Hello World'' displayed in the console. 
	\end{itemize}

\item Now let's go ahead and get the ``ercTools'' repository on your machine. It contains tools for making tables in python and also for drawing datalabels and value labels from stata into Python. 

\item cd \$erc

\item sudo apt-get install git

\item git clone ssh://$<Athens ID>$@athensx.uchicago.edu/share/klmshare/ercTools

\item Now let's add Sublime-Text-2 for our latex editor (also for Python and anything to do with text/code). 

\item sudo add-apt-repository ppa:webupd8team/sublime-text-2

\item sudo apt-get update

\item sudo apt-get install sublime-text

\item sudo apt-get install latexmk

\item Here is a good \href{http://www.practicallyefficient.com/home/2012/11/29/working-with-latex-in-sublime-text-2}{tutorial} for learning LateX editing in Sublime-text-2. 

\item You should be able to run Python scripts immediately from Sublime-Text as well. It is great for smaller, medium size Python projects. 

\end{enumerate}

\end{document}